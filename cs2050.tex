\documentclass[english,openany]{book}
\usepackage[utf8]{inputenc}
\usepackage{babel}
\usepackage{amssymb,amsmath,listings,parskip}
\usepackage{csquotes}
\MakeOuterQuote{"}
\usepackage{hyperref}
\hypersetup{
    colorlinks,
    allcolors=black,
    hidelinks
}

\begin{document}

    \title{Notes from CS 2050: Introduction to Discrete Math for Computer Science}
    \author{Rafael A. O. Paulucci}
    \date{Fall 2018}

    \maketitle

    \tableofcontents

    \lstset{frame=tb,
    showstringspaces=false,
    columns=flexible,
    basicstyle={\small\ttfamily},
    numbers=none,
    breaklines=true,
    breakatwhitespace=true,
    tabsize=4
    }

    \chapter{Proofs (Rosen 1)}

    \textbf{Note: chapter and section numbers are according to the book Discrete Mathematics and its Applications, Kenneth Rosen, 7th edition.}

    \section{Intro to Proofs (Rosen 1.7)}

	\begin{tabular}{c|c|c}
		Technique&Assume&Show\\
		\hline
		direct proof&$p$&$q$\\
		proof by contraposition&$\neg q$&$\neg p$\\
		vacuous proof&(nothing)&$\neg p$ ($p$ is always false)\\
		trivial proof&(nothing)&$q$($q$ is always true)\\
		proof by contradiction&$p \wedge \neg q$&$?$\\
	\end{tabular} \newline

	Prove p $\rightarrow$ q.

	1) Assume p is true.

	2) Work in definitions. Do math, do logic.

	3) q is then also true.\newline

	Definition. An integer $n$ is even if there exists an integer $k$ where $n = 2k$\newline

	Definition. An integer $n$ is odd if there exists an integer $k$ where $n = 2k + 1$\newline

	\subsection{Direct proof}

	\textbf{Direct proof: The sum of two even integers is even.}

	$p \rightarrow q$
	a, b $\in {\mathbb Z}$

	p: a and b are even

	q: a + b is even.

	a and b are even (assume p)

	a = 2k, k $\in {\mathbb Z}$ (by definition even)

	b = 2k', $k' \in {\mathbb Z}$ (by definition even)

	a+b = 2k + 2k' (add LHS, RHS)

	a+b = 2(k+k') (factor)

	Note: by definition of even, a+b is even, having form a+b = 2k' where k'' = k + k', k'' $\in {\mathbb Z}$

	Integers are closed with respect to addition.

	Conclusion: I've shown that assuming $p$ leads to $q$ being true, hence $p \rightarrow q$.\newline

	\textbf{Direct proof: If $n$ is odd, then $n^2$ is odd.}

	Proof by contraposition: If $n$ is an integer and $3n+2$ is odd, then $n$ is odd.

	$p \rightarrow q$

	$p: 3n+2$ is odd

	$q: n$ is odd ($n \in {\mathbb Z}$)

	$\neg$ ($n$ is odd) (assume $\neg q$)

	$n$ is even (by definition of even and odd)

	$n = 2k, k \in {\mathbb Z}$ (by definition of even)

	$3n+2$ Let's examine this.

	$3(2k) + 2$ Substitute for $n$.

	$2(3k+1)$ Factor

	Notice this is $\neg p$, the fact than $3n+2$ is even.

	Matching the format of $2k'$, where $k' = 3k+1, k' \in \mathbb Z$

	Conclusion: since $\neg q$ leads to $\neg p$, I've shown $p \rightarrow q$.

		\textbf{1. If $n = ab$ then $a \leq \sqrt{n} \lor b \leq \sqrt{n}$ with $n, a, b \in \mathbb Z^+$}

	\textbf{2. If $m$ and $n$ are perfect squares, then $mn$ is a perfect square.}

	Definition: The integer $x$ is a perfect square if $x = a^2$, with $x, a \in \mathbb Z$.

	Direct proof:

	\begin{tabular}{c|c}
		Step&Rule\\
		\hline
		$m$ is a perfect square. $n$ is a perfect square. &Assume $p$.\\

		$m = a^2, a \in \mathbb Z$&\\

		$n = b^2, b \in \mathbb Z$&\\

		$mn = a^2 b^2$ &Multiply LHSs + RHSs\\

		$mn = (ab)^2$ &Math.\\
	\end{tabular}

	Hence $mn$ is a perfect square, by definition, where $mn = c^2$, $c=ab$, with $c \in \mathbb Z$. This is our $q$ proposition.

	Conclusion: Assuming $p$ leads to $q$, hence $p \rightarrow q$.\newline

	\textbf{3. $\forall x \exists y \exists z (x=y^2 + z^2)$ with $y,z \in \mathbb Z$ and $x \in \mathbb Z^+$}

	\subsection{Proof by contraposition}

	\begin{tabular}{c|c}
		Step&Rule\\
		\hline
		$\neg(a \leq \sqrt{n} \lor b \leq \sqrt{n})$ & Assume $\neg q$\\

		$a \leq \sqrt{n} \wedge b \leq \sqrt{n}$ & DeMorgan's\\

		$a > \sqrt{n} \wedge b > \sqrt{n}$ & Definiton of $\leq$\\

		$ab > \sqrt{n}\sqrt{n}$ & Multiply LHSs + RHSs\\

		$ab > n$ & Multiply\\
	\end{tabular}

	Therefore, $ab \neq n$ by definition of $>, =$. Note this is $\neg p$.

	Conclusion: I have shown that, assuming $\neg q$ leads to $\neg p$, hence $p \rightarrow q$. \newline

	\subsection{Proof by counterexample}

   (Used to prove a claim is false.)

	See $x=3$.

	$3 \in \mathbb Z^+$ and yet no sum of $y^2$ and $z^2$ will yield $3$.

	Let's look at the smallest perfect squares.

	\begin{tabular}{c}
		$0^2 = 0$\\
		$-1^2 = 1^2$ = 1\\
		$-2^2 = 2^2 = 4$\\
	\end{tabular}

	And these continue in increasing order.

	A sum of 3 could allow some pair of 0 or 1. Yet,

	\begin{tabular}{c}
		$0+0=0$\\
		$0+1=1$\\
		$1+0=1$\\
		$1+1=2$\\
	\end{tabular}

	Any larger sum overshoots 3.

	Conclusion: $x=3$ fails, so this claim is false.\newline


	\subsection{Proof by contradiction}

	Instead of starting from $p \rightarrow q$, we'll start from the following:

	$\neg (p \rightarrow q)$

	Material implication: $\neg(\neg p \lor q)$

	Assume $p \wedge \neg q$. If we find $p \wedge \neg q \wedge q$, there's a contradiction. Likewise for $\neg p \wedge p \wedge \neg q$.

	\section{Some solved exercises}

		\textbf{5. b} (DeMorgan's is done incorrectly)\\

	\textbf{6. a} This is material implication.\\

	\textbf{7. }

	\begin{tabular}{cc|cccc|c}
		p&q&$p \leftrightarrow q$&$\neg(p \leftrightarrow q)$&$\neg q$&$p \leftrightarrow \neg q$&Conclusion\\
		\hline
		T&T&T&F&F&F&T\\
		T&F&F&T&T&T&T\\
		F&T&F&T&F&T&T\\
		F&F&T&F&T&F&T\\
	\end{tabular}

	Conclusion: Since the two columns $\neg(p \leftrightarrow q)$ and $p \leftrightarrow \neg q$ are identical, $\neg(p \leftrightarrow q) \equiv p \leftrightarrow \neg q$.\\

	\textbf{8.}

	\begin{tabular}{c|c|c}
		Step&Statement&Rule\\
		\hline
		5&$p \rightarrow r$&contrapositive of 2.\\
		6&$p$&disjunctive syllogism (4. and 1.)\\
		7&$r$&modus ponens (6. and 5.)\\
		8&$s$&modus ponens (3. and 7.)\\
	\end{tabular}

	$\therefore$ Using the rules of inference and premises, we have shown $s$.\\

	\textbf{9.} $\forall y \exists x_1 \exists x_2 ( M(y) \rightarrow ( (T(x_1, y) \wedge C(x_1)) \wedge (T(x_2, y) \wedge F(x_2))  ))$\\

	\textbf{10.} $\forall x \exists y (F(x) \rightarrow (T(x,y) \wedge M(y)))$\\

	\textbf{11.} $\exists x (F(x) \wedge C(x))$\\

	\textbf{12.} $\exists x_1 \exists x_2  \forall x_3 ( C(x_1) \wedge C(x_2) \wedge (x_1 \neq x_2) \wedge ((x_1 \neq x_3) \wedge (x_2 \neq x_3)) \rightarrow \neg C(x_3))$


	\chapter{Set Theory (Rosen 2)}

	\section{Sets}

	\textbf{Unordered collection of elements.}

	\subsection{Intro to Sets (Rosen 2.1)}

	Example: $\{1,2,3\} = \{2,1,3\} = \{1,1,3,2,3,2\}$\newline

	$S = \{1,3,7,7,$ Mary, Lee$,3.5,1\}$

	The cardinality of $S$ is 6.

	$\mathbb Z^+ = \{ 1,2,3,... \}$

	The cardinality of $\mathbb Z^+$ is $\infty$.

	$B = \{\mathbb Z, \mathbb Z^+, \mathbb R \}$

	The cardinality of $B$ is 3.

	$C = \{ \{\}, \{1,2\},1,2,\{1,1,2 \}, \{\{\}\} \}$

	The cardinality of $C$ is 5. The only element not counted is $\{1,1,2 \}$, because it is equivalent to $\{1,2\}$.

	Note: $1 \neq \{1\}$, and $1 \in \{1\}$

	Also, the set that contains the empty set is not equal to the empty set itself.

	$D = \{\mathbb Z, \mathbb Z^+ \cup \{0\} \cup \mathbb Z^- \}$

	The cardinality of $D$ is 1, because the second element is equivalent to the first.\newline

	\noindent
	\subsection{Set Builder Notation, List Notation}

	$B = \{2,4,6,8,...\}$

	$\mathbb Z^+ = \{ 1,2,3,4,... \}$

	$B = \{ 2x \ | \ x \in \mathbb Z^+\}$ \\

	$A = \{1,2\}$

	$B = \{5,6,7\}$

	$A \times B = \{ (1,5),(1,6),(1,7),(2,5),(2,6),(2,7) \}$

	$A \times B = \{ (a,b) \ |\ a \in A, b \in B \}$\\

	$S = \{2,9,28,65,...\}$

	$S = \{ x^3 + 1 \ | \ x \in \mathbb Z^+ \}$\\

	$A = B \iff \forall x (x \in A \iff x \in B)$\\

	$A \subset B \iff \forall x (x \in A \rightarrow x \in B) \wedge \exists x (x \in B \wedge x \not\in A)$

	(Note: the above is a proper subset. A "regular" subset is denoted by $\subseteq$.)

	$A \subseteq B \iff \forall x (x \in A \rightarrow x \in B)$

	\section{Cardinality (Rosen 2.2)}

	$\{\} = \emptyset$

	So $/\{ \{\}, \emptyset, \{\{\}\} \}/ = 2$\\

	\subsection{Power Set}

	$\mathbb P (S)$ is the set of all subsets of $S$.

	$S = \{\}$, $\mathbb P (S) = \{ \{ \} \}$

	$S = \{1\}$, $\mathbb P (S) = \{  \{\}, \{1\}  \}$

	$S = \{1, 2\}$, $\mathbb P (S) = \{  \{\},  \{ 1 \}, \{ 2 \},  \{1, 2\}  \}$

	\dots

	$/ \mathbb P(S)/ = 2^{ /S/ }$ (bars are notation for set cardinality)\\

	$A \times B = \{ (a,b)\ |\ a \in A \wedge b \in B \}$ (ordered 2-tuple)

	$A \times B \times C = \{ (a,b,c)\ |\ a \in A \wedge b \in B \wedge c \in C \}$ (ordered 3-tuple)

	$A \cup B \{ x \ | \ x \in A \lor x \in B\}$

	$B - A = \{ x \ |\ x \in B \wedge x \notin A \}$

	$\bar{A} = U - A$ (complement of A) ($U$ is a hypothetical universal set in a given context)

	$\bar{A} = \{ x \ |\ x \in U \wedge x \notin A \}$

		Use set builder notation and logical equivalencies to show De Morgan's law for sets holds.

	$\bar{A \cap B} = \bar{A} \cup \bar{B}$

	$\bar{A \cap B} = \{ x \ |\ x \notin A \cap B\} $  (definition of complement)

	$= \{ x \ |\ \neg(x \in A \cap B) \} $ (definition of $\notin$)

	$= \{ x \ |\ \neg(x \in A \wedge x \in B) \} $ (definition of $\cap$)

	$= \{ x \ |\ \neg(x \in A) \lor \neg(x \in B) \} $ (De Morgan's for logic)

	$= \{ x \ |\ (x \notin A) \lor (x \notin B) \} $ (definition of $\notin$)

	$= \{ x \ |\ (x \in \bar{A}) \lor (x \in \bar{B}) \} $ (definition of complement)

	$= \{ x \ |\ x \in \bar{A} \cup \bar{B} \} $ (definition of $\cup$)

	$= \bar{A} \cup \bar{B}$

	$\therefore$ It is true, as shown, that $\bar{A \cap B} = \bar{A} \cup \bar{B}$.\\

	\textbf{Membership table}

	$A \cap (B \cup C) = (A \cap B) \cup (A \cap C)$ Distributive property

	Table indicates if $x$ belongs to set $A, B, C \dots$

	\begin{tabular}{c|c|c|c|c|c|c|c}
		$A$&$B$&$C$&$(B \cup C)$&$A \cap (B \cup C)$&$A \wedge B$&$A \wedge C$&$(A \cap B) \cup (A \cap C)$\\
		\hline
		1&1&1&1&1&1&1&1\\
		1&1&0&1&1&1&0&1\\
		1&0&1&1&1&0&1&1\\
		1&0&0&0&0&0&0&0\\
		0&1&1&1&0&0&0&0\\
		0&1&0&1&0&0&0&0\\
		0&0&1&1&0&0&0&0\\
		0&0&0&0&0&0&0&0\\
	\end{tabular}

	Since for every category of elements the result is the same, $A \cap (B \cup C) = (A \cap B) \cup (A \cap C)$.

	$\cap$ is like multiply.

	$\cup$ is like add.

	\section{Functions}

	Function $f$ from $A$ to $B$ (non-empty sets): $f : A \rightarrow B$

    Examples:\\
    %TODO: Insert function sets diagram

	$f :\ $characters$ \rightarrow $grades (domain to co-domain)

    (The function above is not one-to-one and not onto.)

    abs : $\mathbb Z \rightarrow \mathbb Z$ (range: $\mathbb Z^{\geq 0}$)\\

    %TODO: Example of 1-to-1 but not onto; example of not 1-to-1 but onto; example of 1-to-1 and onto.

    When a function is one-to-one and onto, we can find an inverse function $f^{-1}$.

    $f$ is one-to-one $\leftrightarrow$ $\forall a \forall b (a \neq b \rightarrow f(a) \neq f(b))$, with $a,b \in$ domain of $f$.

    $f$ is onto if $\forall y \exists x (f(x) = y)$, with $x \in$ domain of $f$, $y \in$ co-domain of $f$.\\

    In relational databases:

    $f : A \rightarrow B \subseteq A \times B$

    $A = \{1,2\}$

    $B = \{10,20,30\}$

    $A \times B = \{(1,10),(1,20),(1,30),(2,10),(2,20),(2,30)\}$ (all theoretically possible transitions; Cartesian product.)

    We can define the "real" transitions to be $\{(1,10),(2,10)\}$, using the function.

    \chapter{Algorithms and Complexity (Rosen 3)}

    \section{Intro to Big O}

	Problem: Calculate compound interest.

    Assume $P_0 = 10,000$, interest rate 11\%.

    First technique: Recurrence Equation

    $P_n = P_{n-1} \times 1.11$

    Big O: $O(n)$ (n is the number of years)

    $P_1 = P_0 \times 1.11$

    $\vdots$

    $P_{30} = P_{29} \times 1.11$

    $P_{100} = P_0 \times 1.11^{100}$\\

    "Shortcut" for $\sum_{i=1}^{n} i$ with $O(1)$ complexity (closed form)

    Inverting the order of terms of the sum and adding them to the original sum, we will get $n$ elements. Each element is going to equal $n+1$. Since we added the same sum in the opposite order, we have to divide the final result by 2.

    So, a formula for this kind of sum is  $\sum_{i=1}^{n} i = \frac{n(n+1)}{2}$

    procedure max($a_1, a_2, a_3, ..., a_n$ : integers)

    max := $a_1$

    for 1:= 2 to $n$

    $\quad$ if $a_i > $ max then

         $\quad\quad$ max := $a_i$

    Big O is $O(n)$\\

    \noindent
    function exponential($b$: positive integer, $a$: non-negative integer)

    if $a = 0$ then

    $\quad$ result := 1

    else

    $\quad$ result := b $\cdot$ exponential($b,\ a-1)$

    return result

    Big O is $O(a)$\\

    \noindent
    function fastExpo($b$: positive integer, $a$: non-negative integer)

    if a = 0 then

    $\quad$ result := 1

    else if e \% 2 = 0 then

    $\quad$ half := fastExpo(b, a/2)

    $\quad$ result := half $\cdot$ half

    else

    $\quad$ result := b $\cdot$ fastExpo(b, a-1)

    return result\\

    Example: $3^{12} = 3^6 \cdot 3^6 = g \cdot g$

    Big O is $O(\log n)$

    The fastExpo function is an example of "divide-and-conquer".

    There are also, for instance, dynamic programming (example: longest common subsequence), and greedy algorithm.

    \section{Greedy algorithms}

    \subsection{Complexity of MST and Prim's}

    \textbf{Big O}

    \begin{tabular}{c|c|cc}
        Notation&Time&$n=100$&$n=200$\\
        \hline
        $O(1)$&constant&1&1\\
        $O(\log n)$&log time (base 2)&7&8\\
        $O(n)$&linear&100&200\\
        $O(n \log n)$&&700&1600\\
        $O(n^2)$&&10000&40000\\
        $O(2^n)$&&huge (10 commas)&enormous\\
        $O(n!)$&&(40 commas)&super enormous\\
    \end{tabular}\\

    \subsection{Towers of Hanoi}

    \begin{lstlisting}
    def towers(n, a, b, c):
        if n == 1:
            print("Move disk from %s to %s"%(a,b))
        else:
            towers(n-1, a, c, b)
            towers(1, a, b, c)
            towers(n-1, c, b, a)
    \end{lstlisting}

    \textit{Recurrence equation:}

    Time(1): 1 move

    Time(n): Time$(n-1)$ + Time(1) + Time$(n-1)$

    Closed form: $O(2^n) - 1$

    \chapter{Modular Arithmetic (Rosen 4)}

    \section{Congruency (Rosen 4.1)}

    If $a, b \in \mathbb Z$ and $m \in \mathbb Z^+$, then $a$ is congruent to $b$ modulo $m$, if $m$ divides $a-b$.

    We write $a \equiv b\ (\mod m)$.\\

    Examples:

    $7 \equiv 13 (\mod 6)$, $7 - 13 = -6$, $6 | -6 ?$ (yes, multiplier $-1$)

    $3 \not\equiv 10 (\mod 2)$, $10 - 3 = 7$, $2 \not| 7 ?$, $3 \mod 2 \neq 10 \mod 2$\\

    Let $a,b \in \mathbb Z, m \in \mathbb Z^+$. Then, $a \equiv b (\mod m) \leftrightarrow \exists k \in \mathbb Z$ s.t. $a = b + km$.

    If $a \equiv b (\mod m)$ and $c \equiv d (\mod m)$, then $a + c \equiv b + d (\mod m) ?$

    By definition of congruency: $a = b + km, k \in \mathbb Z$

    $c = d + k^{'} m, k \in \mathbb Z$

    Add LHSs, RHSs: $a+c = b + km + d + k^{'} m$

    Regroup, refactor: $a + c = b + d + (k + k^{'})m$

    By definition of congruency modulo $m$, $a + c \equiv b + d (\mod m)$.

    Conclusion: yes, the statement above is true.

    $a \equiv b (\mod m) \wedge c \equiv d (\mod m) \rightarrow a + c \equiv b + d (\mod m)$ as shown, with $a,b,c,d \in \mathbb Z$ and $m \in \mathbb Z^+$.

    \section{Binary Numbers (Rosen 4.2)}

    $123_{10} = 1 \cdot 10^2 + 2 \cdot 10^1 + 3 \cdot 10^0$

    $10110_2 = 1 \cdot 2^4 + 0 \cdot 2^3 + 1 \cdot 2^2 + 1 \cdot 2^1 + 0 \cdot 2^0 = 22_{10}$


    $1011011_2 = 133_8 = 5B_{16}$

    (convert in blocks of 3 bits for octal, and in blocks of 4 bits for hexadecimal)\\

    \textbf{Example:} Convert $241_{10}$ to binary.

    241 div 2 = 120, remainder 1

    120 div 2 = 60, remainder 0

    60 div 2 = 30, remainder 0

    30 div 2 = 15, remainder 0

    15 div 2 = 7, remainder 1

    7 div 2 = 3, remainder 1

    3 div 2 = 1, remainder 1

    1 div 2 = 0, remainder 1

    Read from bottom to top: $11110001_2$

    \section{Primes, GCD, LCM (Rosen 4.3)}

    \textbf{Prime:} integer greater than 1 that is divisible by only 1 and itself

    Every positive integer $>1$ is divisible at least by 1 and itself.

    \textbf{Composite:} positive integer $>1$ that is not prime.

    Integer $n$ is composite $\iff \exists a \in \mathbb Z$ s.t. $a | n$ and $1 < a < n$

    \subsection{Fundamental Theorem of Arithmetic}

    Every positive integer $>1$ can be written uniquely as a prime number or as the product of two or more primes, where the prime factors are written in non-decreasing order.

    Examples: $100 = 2 \cdot 2 \cdot 5 \cdot 5$, $641 = 641$, $999 = 3^3 \cdot 37$

    Longer example: $7007$

    $2|7007$ fails

    $3|7007$ fails

    $5|7007$ fails

    $7|7007$ works: $7007/7 = 1001$

    $7|1001$ works: $1001/7 = 143$

    $7|143$ fails

    $11|143$ works: $11/143 = 13$

    $7007 = 7 \cdot 7 \cdot 11 \cdot 13$

    The factorization can be optimized, for example, by only verifying for primes up to the square root of the original number.\\

    \subsection{Greatest Common Divisor (GCD)}

    gcd(36, 24) = 12

    $36 = 2^2 \cdot 3^2$, $24 = 2^3 \cdot 3^1$

    What 36 and 24 have in common at most, prime by prime, is $2^2 \cdot 3^1 = 12$ (minimized exponents).

    gcd($2^{13} \cdot 3^5 \cdot 7^1 \cdot 13^2,\ 2^2 \cdot 3^1 \cdot 11^5$) = $2^2 \cdot 3^1$

     \textbf{Co-prime or relatively prime}

     Example: $11^2 \cdot 13^5$ and $3^2 \cdot 5^3$. The gcd is 1.

     \subsection{Least Common Multiple (LCM) and Euclid's Algorithm}

     lcm($2^{13} \cdot 3^5 \cdot 7^1 \cdot 13^2, 2^2 \cdot 3^1 \cdot 11^5$) = $2^{13} \cdot 3^5 \cdot 7^1 \cdot 11^5 \cdot 13^2$

     lcm(120, 500) = lcm$(2^3 \cdot 3 \cdot 5,\ 2^2 \cdot 5^3) = 2^3 \cdot 3^1 \cdot 5^3 = 3000$\\

    gcd(630, 196)

    630 mod 196 = 42

    gcd(196, 42)

    196 mod 42 = 28

    gcd(42, 28)

    42 mod 28 = 14

    gcd(28, 14)

    28 mod 14 = 0

    gcd(14,0)

    So gcd(630, 196) = 14

    \chapter{Induction (Rosen 5)}

    \section{Mathematical Induction (Rosen 5.1)}

    \subsection{Proofs by Induction}

    \subsubsection{Ladder}

    Can we reach (prove) the first rung? (i.e., the smallest problem size; the basis step, base case, $P(1)$).

    Inductive step: $k,\ k+1$ rungs.

    $P(k) \rightarrow P(k+1)$. Assume $P(k)$, don't prove it.

    Conclusion: since the basis step and induction step have been shown to be true, \textit{by the principle of mathematical induction}, $\forall n P(n), n \in \mathbb Z^+$ is true. \\

    $P(n): \sum_{i=1}^{n} i = \frac{n(n+1)}{2}, \forall n P(n), n \in \mathbb Z^+$.

    \textbf{Proof by mathematical induction.}

    Basis step: $P(1): \sum_{i=1}^{n} i = 1$ The summation (LHS) yields 1.

    $\frac{1(1+1)}{2} = 1$ The RHS (closed form) yields 1 when simplified.

    Conclusion: since the sum is 1 and the closed form is 1, $P(1)$ is true.

    Inductive step:

    I will prove $P(k) \rightarrow P(k+1), k \in \mathbb Z^+$.

    Assume $P(k): \sum_{i=1}^{k} i = \frac{k(k+1)}{2}$.

    \noindent\fbox{
        \parbox{\textwidth}{
            \textit{Aside:}

            $1+2+3+4+\dots+k \rightarrow k(k+1)/2$

            To add the first $k+1$ numbers, $1+2+3+4+\dots+k+[k+1] \rightarrow k(k+1)/2 + [k+1] = \sum_{i=1}^{k+1} i = \frac{(k+1)[(k+1)+1]}{2}$
        }
    }

    $\sum_{i=1}^{k} i + [k+1] = \frac{k(k+1)}{2} + [k+1]$

    $\sum_{i=1}^{k+1} i + [k+1] = \frac{k(k+1)+2[k+1]}{2}$ Simplify the summation, find common denominator.

    \qquad \qquad$= \frac{(k+1)(k+2)}{2} = \frac{(k+1)((k+1)+1)}{2}$

    Note this is our $k+1$ statement. Hence, it does follow from our $k$ statement. I have shown $P(k) \rightarrow P(k+1)$, completing the inductive step.

    Conclusion: since the basis step and inductive step are both true, by the principle of mathematical induction, $P(n)$ is true for all positive integers.\\

    \subsubsection{Sum of $n$ elements}

    $1+3+5+7+\dots =\ ?,\ n \in \mathbb Z^+$

    The sum of the first element is 1.

    The sum of the first 2 elements is 4.

    The sum of the first 3 elements is 9.

    It seems like the sum of the first $n$ elements is $n^2$.

    $P(n): \sum_{i=1}^{n} (2i-1) = n^2,\ n \in \mathbb Z^+$ (conjecture)

    \textbf{Proof by mathematical induction.}

    Basis step: $P(1):$ $\sum_{i=1}^{1} (2i-1) = 1^2$

    LHS: $2(1) - 1 = 2-1 = 1$

    RHS: $1^2 + 1$

    Since LHS = RHS, $P(1)$ is true.

    Inductive step:

    We will show $P(k) \rightarrow P(k+1), k \in \mathbb Z^+$

    $P(k): \sum_{i=1}^{k} (2i-1) = k^2$ Assume $P(k)$.

    $\sum_{i=1}^{k} (2i-1) + [2(k+1)-1] = k^2 + [2(k+1)-1]$ Add next term to both

    $\sum_{i=1}^{k+1} (2i-1) = k^2 + [2(k+1)-1]$ Clean up sum

    \qquad \qquad $= k^2 + 2k + 2 - 1$ Simplify RHS

    \qquad \qquad $= k^2 + 2k + 1$ Simplify

    \qquad \qquad $= (k+1)^2$ Factor

    $P(k)$ does lead to $P(k+1)$, being true, hence, $P(k) \rightarrow P(k+1)$

    Conclusion: since the basis step and inductive step are both true, by the principle of mathematical induction, $P(n)$ is true for all positive integers.\\

    \subsubsection{Divisors}

    $P(n): 3|(n^3 - n) \forall n,\ n \in \mathbb Z^+$

    Proof by math induction.

    Base step, $P(1)$:

    Let's look at $P(1): 3|(1^3 - 1)$

    $3|(1-1)$ (simplify)

    $3|0$ (simplify)

    By definiton, 3 divides 0, since $3 \cdot 0$ yields 0. Therefore, $P(1)$ is true.\\

    Inductive step: prove $P(k) \rightarrow P(k+1), k \in \mathbb Z^+$.

    Let's look at $P(k+1)$.

    $P(k+1) : 3|[(k+1)^3 - (k+1)]$

    $3|(k^3+3k^2+3k+1)-(k+1)$ Expand

    $3|(k^3 + 3k^2 + 3k - k)$ Regroup

    $3|(k^3 - k + 3k^2 + 3k)$ Regroup

    $3|[(k^3-k)+3(k^2+k)]$ Factor 3

    Note that by our inductive hypothesis, $P(k) (k^3-k)$ is divisible by 3. We conclude that $3(k^2+k)$ is divisible by 3 since it is a multiple of 3, $k^2 + k \in \mathbb Z$.

    The sum of two values divisible by 3 yields a sum divisible by 3.

    I've shown $P(k) \rightarrow P(k+1)$.

    Conclusion: since the base step and inductive step are both true, $P(n)$ is true $\forall n \in \mathbb Z^+$, by the principle of mathematical induction.\\

    \subsubsection{Tiling}

    Problem. $P(n)$: let $n \in \mathbb Z^+$. Show that every 2 by 2 board with one square removed can be tiled by right triangles.

    Proof by mathematical induction.

    Base case. $P(1): $ 2 by 2 board can be tiled having any one square removed as shown below.

    %TODO: add illustration

    Base case is complete.

    Inductive step. $P(k) \rightarrow P(k+1), k \in \mathbb Z^+$

    Let's look at a $2^{k+1} \times 2^{k+1}$ board.

    %TODO: add illustration

    \section{Strong Induction (Rosen 5.2)}

    Recall the "ladder" from the mathematical induction section.

    With strong induction, it is possible to use more than 1 previous result to reach a desired next result.

    Take care not to go outside of your domain when going back

    \subsection{Postage stamps}

    %TODO: Complete the beginning of this demo

    We have 4 cent and 5 cent stamps.

    We have $3 \times 4$ cent stamps, or 12 cents total.

    Inductive step: will show $\forall j (12 \leq j \leq k) P(j) \rightarrow P(k+1)$

    Let's look at $k + 1$ postage.

    Use the postage for $k - 3$ (which is doable by our inductive hypothesis) and add a 4 cent stamp to it.

    Note $k - 3 + 4 = k + 1$. Hence, $P(k+1)$ is also true.

    \newpage

    \section{Recursion (Rosen 5.3)}

    \subsection{Factorials}

    Factorials $n!,\ n \in \mathbb Z^{\geq 0}$

    Base case: $0! = 1$

    Recursive step: $(n+1)! = (n+1) \cdot n!$\\

    \subsection{Fibonacci}

    Fibonacci ($n \in \mathbb Z^{\geq 0}$)

    Base case: Fib(0) = 0

    Fib(1) = 1

    Recursive step: Fib(n+2) = Fib(n) + Fib(n+1)\\

    Base case: $3 \in S$

    Recursive step:
    \qquad if $x \in S$ and $y \in S$, then $x+y \in S$

    List notation: $S =\{3, 6, 9, 12, 15, \dots\} $

    Set builder: $S = \{3x |\ x \in \mathbb Z^+\}$\\

    \subsection{Alphabets and Strings}

    $\Sigma$ is your alphabet. Examples: $\Sigma = \{0,1\}$  or $\Sigma = \{A,T,C,G\}$, and so on.

    $\Sigma^*$ (Kleene closure) is the set of all strings that can be built from symbols in $\Sigma$. $\lambda$ is an empty string.

    Example: $\Sigma = \{0,1\}$

    $\Sigma^* = \{\lambda, 0, 1, 00, 01, 10, 11, 000, 001, 010, 011, 100, 101, 110, 111, 0000, \dots \}$

    Recursive definition of $\Sigma^*$.

    Base case: $\lambda \in \Sigma^*$

    Recursive step: If $w \in \Sigma^*$ and $x \in \Sigma$, then $wx \in \Sigma^*$ (concatenation)

    We will notice that the elements 0 and 1, in the previous example, are actually formed by $\lambda0$ and $\lambda1$

    $w$ is a string over alphabet $\Sigma$: $w \in \Sigma^*$

    Define the reverse, $w^R$, recursively.

    Base case: $\lambda^R = \lambda$

    Recursive step: If $w \in \Sigma^*$ and $x \in \Sigma$, then $(wx)^R = x(w)^R$ (i.e, put the last character in the front position of the string)

    Set of all palindromes over $\Sigma$. Call the set $P$.

    Recursive definition:

    Base case: $\lambda \in P$. If $x \in \Sigma, x \in P$.

    Recursive step: If $w \in P$ and $x \in \Sigma$, then $xwx \in P$.

    \subsubsection{Length of a string}

    Basis step:

    length$(\lambda) = 0$

    Recursive step:

    if $w \in \Sigma^*$ and $x \in \Sigma$, then length($wx$) = length($w$) + 1\\


    $\Sigma^{even}$: \textit{even-length strings over} $\Sigma$.

    Basis step:

    $\lambda \in \Sigma^{even}$

    Recursive step:

    if $w \in \Sigma^{even}$ and $x \in \Sigma$, $y \in \Sigma$, then $wxy \in \Sigma^{even}$

    \subsection{Palindromes}

    Base case: $\lambda \in P$ if $x \in \Sigma$ then $x \in P$.

    Recursive: if $x \in \Sigma$ and $w \in P$.

    then $xwx \in P$

    (note: this doesn't cover everything: $xxwxx \in P$)\\


	\chapter{Counting (Rosen 6)}

    \section{Counting (Rosen 6.1)}

    \subsection{Product rule}

    1 dollar: cupcake

    1 type of cake, 1 type of icing always.

    Cake: \{chocolate, blueberry, red velvet, peach, strawberry, vanilla\} (6 types)

    Icing: \{mango, vanilla, cream cheese, coffee\} (4 types)

    Product rule: $6 \cdot 4 = 24$\\

    \subsection{Sum rule}

    3 dollars: pizza

    1 topping, sauce, crust always.

    Crust: \{thin, deep dish\} (2 types)

    Sauce: \{tomato, pesto, alfredo, BBQ\} (4 types)

    Topping: \{pineapple, pepperoni, mushrooms\} (3 types)

    (Extended) product rule: $2 \cdot 4 \cdot 3 = 24$\\

    1 free cupcake $\oplus$ 1 free pizza

    How many choices? Sum rule: $24 + 24 = 48$\\

    \textbf{Product and sum rules combined}

    12 offices (empty and unique)

    2 interns (Zim and Gir, unique)

    How many ways to have office assignments?

    If they must share an office: 12 possibilities.

    If they must NOT share an office: $12 \cdot 11$ possibilities.

    If sharing doesn't matter: $12 + 12 \cdot 11$ (there is no way to share and not share at the same time)\\

    \subsection{Extra examples}

    \subsubsection{Car tag}

    Format: 3 letters and 4 numbers (LLLDDDD).

    26 possible letters, 10 possible numbers.

    Possibilities: $26^3 \cdot 10^4$

    If we want to avoid one certain sequence of 3 starting characters (that might be obscene), we would have $26^3 \cdot 10^4 - 10^4$ possibilities (inclusion-exclusion technique). Note we are not taking into account the spelling itself.\\

    How many bit strings are of length 7? $2^7$\\

    \subsubsection{BASIC language variable names}

    Rules:

    1. Starts with a letter (one of 26 letters), and is otherwise alphanumeric (not case-sensitive).

    2. Maximum length is 2.

    3. Cannot use a reserved word. There are 5 "collisions" (reserved words of length 2).

    $V = V_1 + V_2$

    $V_1 = 26$

    $V_2 = 26 \cdot (26+10)$

    so $V = 962$

    Removing the 5 reserved word collisions, $V_f = 957$\\

    \subsubsection{Passwords (Inclusion/Exclusion)}

    Length 6 to 8, alphanumeric: $P = P_6 + P_7 + P_8 = 36^6 + 36^7 + 36^8$

    $P_6 = (26+10)^6 = 36^6$

    If it must contain at least one digit: calculate $P_n$, but subtract all the possibilities that do not contain at least one digit (i.e, the ones that are all letters). \textit{Inclusion/exclusion.}

    $P_{6*} = P_6 - (26)^6$

    $P_{7*} = P_7 - (26)^7$

    $P_{8*} = P_8 - (26)^8$\\

    \subsubsection{Bit strings}

    Number of bit strings with length 8 that start with a 1 or end with 00 (inclusive OR).

    Number of strings that start with 1 + Number of strings that end with 00 $-$ number of strings that start with 1 and end with 00 (to avoid double counting)

    $(1 * 2^7) + (2^6 * 1 * 1) - (1 * 2^5 * 1 * 1)$

    If the OR was exclusive, we would subtract $2^5$ twice.\\

    Bit strings of length 4 without consecutive 1s.

    $\{0000, 0001, 0010, 0100, 0101, 1000, 1001, 1010\}$

    Use a tree with tree pruning to find possible strings.

    \newpage
    \section{Permutation and Combination (Rosen 6.3)}

    We have a set $S$ of size $n$ and we will pick $r$ elements.

    $n = /S/ = 3$

    $S = \{a,b,c\}$

    There are two ways this can go.

    Imagine we want 2 of these. What do we want?

    $/ \{ \{a,b\}, \{a,c\}, \{b,c\} \} / = 3$

    If the order matters:

    $/ \{   <a,b>, <b,a>, <a,c>, <c,a>, <b,c>, <c,b>  \} /  = 6$

    Choose $r$ from $n$ elements.

    If the order of picking matters: \textit{permutation}. If not: \textit{combination}.\\

    \subsection{Permutation}

    We have 100 runners. What is the number of winning permutations for first, second and third place? Everybody finishes, and there are no ties.

    $100 \cdot 99 \cdot 98$

    Number of ways for the top 20:

    $P(100,20) = \frac{100!}{(100-20)!}$

    \subsubsection{Number of permutations of length $r$ from a set of $n$ elements}

    \noindent\fbox{
        \parbox{\textwidth}{
             $$P(n,r) = \frac{n!}{(n-r)!}$$
        }
    }

    \subsubsection{Non-trivial permutation}

    $n = 4$ students

    $S = \{a,b,c,d \}$

    Take 2 students to a competition.

    $C(4,2) = / \{ \{a,b\}, \{a,c\}, \{a,d\}, \{b, c\}, \{b,d\}, \{c,d\} \} / = 6$

    There is no trivial way to "fill in the blanks" as in previous examples.

    Trying to to $4 \cdot 3 \cdot 2 \cdot 1$ would result in $4!/2!$, but this has to be "un-permuted" to avoid double-counting.

    Final result: $\frac{4!}{2! \cdot 2!}$\\

    \subsection{Combination}

    There are 40000 runners, and 100 will be invited to an event. The order of invites does not matter.

    $C(40000,100) = \frac{40000!}{(40000 - 100)! \cdot 100!}$\\

    \subsubsection{Number of combinations of length $r$ from a set of $n$ elements}
        \noindent\fbox{
        \parbox{\textwidth}{
             $$C(n,r) = \frac{n!}{(n-r)! \cdot r!}$$
        }
    }
    Note: $C(100,99) = C(100,1)$ (edge case)\\

    \subsubsection{Bit strings in permutations}

    Number of bit strings of length 8 having exactly six 1s.

    $n = 8,\ r = 6$

    $C(8,6) = \frac{8!}{(8-6)! \cdot 6!}$ (the order of picking the bits themselves does not matter)

    \newpage
    \section{Generalized Permutations and Combinations (Rosen 6.5)}

    How many strings of length 10 are there, considering 26 letters (A-Z)?

    By the product rule, $26^{10}$.\\

    \subsection{"Stars and Bars"}

    3 types of fruit. Pick 4 pieces of fruit ($r=4$).

    Apples, oranges and pears (all treated equal to each other within their group.)

    We have "infinite slices", and we can put the names of fruit alphabetically and represent each possible choice as a bitstream.

    For example, a choice of no apples, no oranges, and 4 pears is 110000. The 1s represent "dividers" and the 0s represent fruit choices.

    011000 is 1 apple, 0 oranges, 3 pears.

    How many bitstrings of length 6 exist containing 4 zeros?

    $r = 4,\ n = 3$ types

    $C(n+r-1, r) = C(3+4-1, 4) = C(6,4)$\\

    \subsubsection{Example: dollar bills}

    Take 4 bills from a cash register. There are slots for \$20, \$10, \$5 and \$1.

    $C(4+4-1, 4) = C(7,4)$ ways to do this.\\

    \subsubsection{Example: cookies}

    4 types of cookies, a customer grabs 6 cookies to go.

    A customer's bag must have all flavors.

    $C(4+2-1, 2) = C(5,2)$.\\

    \subsubsection{Example: sum}

    $a+b+c = 13$, with $a,b,c \in \mathbb Z^{\geq 0}$

    $13 + 0 + 0 = 13$

    $\vdots$

    $0 + 0 + 13 = 13$

    $C(15,13)$ ways to do this. (3 [types of variables] + 13 [sum] $- 1$)

    If we change to $a,b,c \in \mathbb Z^+$, we have $C(3 + 10 - 1,\ 10) = C(12,10)$. We start with 3 and try to find which variables can add 10 more.

    \newpage
    \section{Permutations with Indistinguishable Objects (Rosen 6.5)}

    \subsection{Words with repeated letters}

    In the word SUCCESS, we have 3 equal S characters and 2 equal C characters.

    This results in $\frac{7!}{3!2!1!1!}$ possible arrangements.

    In the word MISSISSIPPI, we have 4 S, 4 I, 2 P.

    This results in $\frac{11!}{4!4!2!}$ possible arrangements.\\

    \subsection{Card game}

    We have 4 unique card players. We deal a hand of 5 cards to each, from a 52-card deck of unique cards.

    This results in $\frac{52!}{5!5!5!5!32!}$ possible arrangements, or $C(52,5) \cdot C(47,5) \cdot C(42,5) \cdot C(37,5)$\\

    \subsection{Basketballs}

    We have 10 indistinguishable basketballs and 8 unique SUVs. It is possible to fit all 10 basketballs in a single SUV.

    Using the Stars and Bars technique,

    SUVs ("types"): A B C D E F G H

    \subsection{Bit strings}

    00000000001111111, or 11100000111100000, and so on.

    $C(17,10)$

    \chapter{Probability (Rosen 7)}

    \section{Finite Probability (Rosen 7.1)}

    \subsection{Definitions}

    \textit{Experiment:} process yielding one of a set of possible outcomes.

    \textit{Sample space:} set of possible outcomes

    \textit{Event space:} subset of sample space

    \subsection{Examples}

    \subsubsection{Socks}

    4 orange socks, 5 blue socks in a drawer.

    What is the probability of retrieving an orange sock?

    P(orange) = $\frac{4}{9}$ = $\frac{/E/}{/S/}$\\

    \subsubsection{Dice}

    Roll 2 dice (6-sided)

    Rolling $<1,2>$ is different than rolling $<2,1>$. There are 36 possible outcomes.

    The possible pairs that add up to 7 are $<1,6>, <2,5>, <3,4>, <4,3>, <5,2>, <6,1>$. There are 6 favourable outcomes.

    P(total = 7) = $\frac{6}{36} = \frac{1}{6}$

    The possible pair that adds up to 12 is $<6,6>$.

    P(total = 12) = $\frac{1}{36}$\\

    \subsubsection{Lottery}

    First prize: pick 4 correct digits. Probability is $\frac{1}{10^4}$ (10 ways for each digit)

    Second prize: pick 3 out of 4 correct digits (value and placement).

    There are 9 ways to "miss" each digit, and 4 possible positions for each digit. Probability is $\frac{36}{10^4}$\\

    \subsubsection{Lottery (again)}

    Let $S$ be the set of positive integers from 1 to 40, just like 40 distinct, numbered balls.

    We will get a subset of 6 numbers with no replacement. To win the game, you must have the 6 correct numbers.

    The probability to win is $\frac{1}{C(40,6)} \approxeq 0.00000026$.

    If replacement was allowed, this would become a Stars and Bars problem.

    The probability would be $\frac{1}{C(40+6-1, 6)} = \frac{1}{C(45,6)}$\\

    \subsubsection{Poker}

    52 cards: 13 ranks $\{2,3,4,5,...,10,J,Q,K,A\}$ $\times$ 4 suits $\{$Spades, Hearts, Clubs, Diamonds$\}$.

    Probability("4 of a kind" given 5-hand card):

    $$\frac{C(13,1) \cdot C(4,4) \cdot C(48,1)}{C(52,5)} = \frac{13 \cdot 48}{C(52,5)}$$\\

    "Full House": 3 of a kind and 2 of a kind (notice that $3Q + 2K \neq 3K + 2Q$)

    $$\frac{C(13,1) \cdot C(4,3) \cdot C(12,1) \cdot C(4,2)}{C(52,5)}$$

    An alternative way to write the numerator is $P(13,2) \cdot C(4,3) \cdot C(4,2)$.\\

    \subsubsection{Number sequence}

    Pick 5 numbers from [1, 50], with no replacement, and order mattering.

    Probability: $\frac{1}{P(50,5)}$
    
    \subsection{Monty Hall Problem}
    
    Problem named after host Monty Hall from the game show Let’s Make a Deal.
    
    3 identical doors (A, B, C). 2 of them have a goat behind, and one has an all-inclusive trip around the world.
    
    Steps:

    1. Player picks 1 door

    2. Monty Hall opens one other door (he knows what the doors have behind, so he always opens a goat door)
    
    3. Player can keep their previous choice or switch to the other closed door.

    WLOG, let’s pretend the player picks door A. The probability of getting the good prize is 1/3. This probability will remain as long as the prize location has not been revealed. So, initially, there is a 2/3 chance the player will get a bad prize.

    After a goat door is opened, the 2/3 chance is “concentrated” in the unopened door. If the player does not change doors, they will have a 1/3 chance of winning. If they change, the chance will go up to 2/3.

    An extension of this problem involves a variation with one million dud doors, and only one correct door. If the host shows the player all duds but one, the probability $\frac{999999}{1000000}$ is “concentrated” like in the original 1000000 problem - i.e., the player should switch to “the best of all the other doors.”

    \chapter{Out-of-book}

    \section{Some Practice Exam 4 (Fall 2018) Solutions}

    1. Basis step:

    $f(0) = 6$

    $f(1) = 10$

    Recursive step: $f(n+2) = f(n+1) + f(n)$\\

    2. Shortest members: $\{1,111,11111,1111111,...\}$\\

    3. Given $f(0) = 3$, $f(n) = 2 \cdot f(n-1) + 6$

    $f(1) = 12;\ f(2) = 30$\\

    4. $\Sigma = \{0,1\}$

    Base case: $\lambda \in P,\ 0 \in P, 1 \in P$

    Recursive step: if ($x \in P$ and $w \in \Sigma$) $\rightarrow wxw \in P$\\

    5. Base case: $1 \in S$

    Recursive step: $x \in S \rightarrow x/3 \in S$\\

    6. flipbits($\lambda$) = $\lambda$

    flipbits(0) = 1; flipbits(1) = 0\\

    Recursive step: $\Sigma = \{0,1\}$. $\Sigma^*$ is the Kleene closure (all possible strings that can be formed from the alphabet $\Sigma$)

    (b $\in \Sigma$ and $x \in \Sigma^*$)

    flipbits(bx) = flipbits(b) + flipbits(x)\\

    \section{Solutions to some problems from Rosen 6.5}

    1. TATTLETALES

    Permutations: $\frac{11!}{4!2!2!2!}$\\

    2. 200 students, 4 houses. Using \textit{Stars and bars}, we have $C(200+(4-1), (4-1)) = C(203, 200) = C(203, 3)$ possible ways to assign students to houses.\\

    3. (a) Equation $a_1 + a_2 + a_3 + a_4 + a_5 + a_6 = 100$. Using \textit{Stars and bars}, we have $C(100+(6-1), (6-1)) = C(105, 5)$ possible ways to solve it.

    (b) If we reqire $a_1, a_2 \geq 2$, then we have $C(96+(6-1), (6-1)) = C(101,5) = C(101, 96)$ ways.

    (c) Equation $a_1 + a_2 + a_3 + a_4 + a_5 + a_6 \leq 100$, where $a_n$ are in $\mathbb Z^{\geq 0}$. We have $C(100+(7-1), (7-1)) = C(106,6)$ ways. There is a "hidden" category created.\\

    \section{Material from Recitation Worksheet 2018-11-28}

    \textbf{7. Laundry:} 5 T-shirts, 6 shorts, 3 dress pants, 4 shirts.

    $5!6! + 3!4!$

    $5 \cdot 6 + 3 \cdot 4 = 42$\\

    \textbf{8. String}

    CABDEFGH

    Treat CAB as one letter and permute. This results in $6!$.\\

    \textbf{9. Bit string}

    Length 8, twice as many 0s than 1s. The answer is 0.

\end{document}
