\documentclass{exam}
\usepackage{amssymb,amsmath,listings,parskip}
\begin{document}
	
    \section{4.1}
    
    If $a, b \in \mathbb Z$ and $m \in \mathbb Z^+$, then $a$ is congruent to $b$ modulo $m$, if $m$ divides $a-b$.
    
    We write $a \equiv b\ (\mod m)$.\\
    
    Examples:
    
    $7 \equiv 13 (\mod 6)$, $7 - 13 = -6$, $6 | -6 ?$ (yes, multiplier $-1$)
    
    $3 \not\equiv 10 (\mod 2)$, $10 - 3 = 7$, $2 \not| 7 ?$, $3 \mod 2 \neq 10 \mod 2$\\
    
    Let $a,b \in \mathbb Z, m \in \mathbb Z^+$. Then, $a \equiv b (\mod m) \leftrightarrow \exists k \in \mathbb Z$ s.t. $a = b + km$.
    
    If $a \equiv b (\mod m)$ and $c \equiv d (\mod m)$, then $a + c \equiv b + d (\mod m) ?$
    
    By definition of congruency: $a = b + km, k \in \mathbb Z$
    
    $c = d + k^{'} m, k \in \mathbb Z$
    
    Add LHSs, RHSs: $a+c = b + km + d + k^{'} m$
    
    Regroup, refactor: $a + c = b + d + (k + k^{'})m$
    
    By definition of congruency modulo $m$, $a + c \equiv b + d (\mod m)$.
    
    Conclusion: yes, the statement above is true. 
    
    $a \equiv b (\mod m) \wedge c \equiv d (\mod m) \rightarrow a + c \equiv b + d (\mod m)$ as shown, with $a,b,c,d \in \mathbb Z$ and $m \in \mathbb Z^+$.
    
    \section{4.2: Binary Numbers}
    
    $123_{10} = 1 \cdot 10^2 + 2 \cdot 10^1 + 3 \cdot 10^0$
    
    $10110_2 = 1 \cdot 2^4 + 0 \cdot 2^3 + 1 \cdot 2^2 + 1 \cdot 2^1 + 0 \cdot 2^0 = 22_{10}$
    
\end{document}