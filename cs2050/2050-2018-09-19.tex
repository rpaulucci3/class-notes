\documentclass{exam}
\title{CS 2050 Class Notes for 2018-09-19}
\author{Rafael A. Okiishi Paulucci}
\usepackage{amsmath,amssymb}
\date{}

\begin{document}
	\maketitle
	\section{Sections 2.1 and 2.2}
	
	$\{\} = \emptyset$
	
	So $/\{ \{\}, \emptyset, \{\{\}\} \}/ = 2$\\
	
	\textbf{Power Set}
	
	$\Bbb P (S)$ is the set of all subsets of $S$.
	
	$S = \{\}$, $\Bbb P (S) = \{ \{ \} \}$
	
	$S = \{1\}$, $\Bbb P (S) = \{  \{\}, \{1\}  \}$
	
	$S = \{1, 2\}$, $\Bbb P (S) = \{  \{\},  \{ 1 \}, \{ 2 \},  \{1, 2\}  \}$ 
	
	\dots
	
	$/ \Bbb P(S)/ = 2^{ /S/ }$ (bars are notation for set cardinality)\\ 
	
	$A \times B = \{ (a,b)\ |\ a \in A \wedge b \in B \}$ (ordered 2-tuple)
	
	$A \times B \times C = \{ (a,b,c)\ |\ a \in A \wedge b \in B \wedge c \in C \}$ (ordered 3-tuple)
	
	$A \cup B \{ x \ | \ x \in A \lor x \in B\}$
	
	$B - A = \{ x \ |\ x \in B \wedge x \notin A \}$
	
	$\bar{A} = U - A$ (complement of A) ($U$ is a hypothetical universal set in a given context)
	
	$\bar{A} = \{ x \ |\ x \in U \wedge x \notin A \}$
	
	
	
	

	
\end{document}