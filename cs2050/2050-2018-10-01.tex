\documentclass{exam}
\usepackage{amssymb,amsmath}
\begin{document}
	
	Problem: Calculate compound interest.
    
    Assume $P_0 = 10,000$, interest rate 11\%.
    
    First technique: Recurrence Equation
    
    $P_n = P_{n-1} \times 1.11$
    
    Big O: $O(n)$ (n is the number of years)
    
    $P_1 = P_0 \times 1.11$
    
    $\vdots$
    
    $P_{30} = P_{29} \times 1.11$
    
    $P_{100} = P_0 \times 1.11^{100}$\\
    
    "Shortcut" for $\sum_{i=1}^{n} i$ with $O(1)$ complexity (closed form)
    
    Inverting the order of terms of the sum and adding them to the original sum, we will get $n$ elements. Each element is going to equal $n+1$. Since we added the same sum in the opposite order, we have to divide the final result by 2.
    
    So, a formula for this kind of sum is  $\sum_{i=1}^{n} i = \frac{n(n+1)}{2}$
    
    \section{Section 3. Algorithms}
    
    procedure max($a_1, a_2, a_3, ..., a_n$ : integers)
    
    max := $a_1$
    
    for 1:= 2 to $n$
    
    $\quad$ if $a_i > $ max then
    
         $\quad\quad$ max := $a_i$
         
    Big O is $O(n)$\\
    
    \noindent     
    function exponential($b$: positive integer, $a$: non-negative integer)
    
    if $a = 0$ then 
    
    $\quad$ result := 1
    
    else
    
    $\quad$ result := b $\cdot$ exponential($b,\ a-1)$
    
    return result
    
    Big O is $O(a)$\\
    
    \noindent
    function fastExpo($b$: positive integer, $a$: non-negative integer)
    
    if a = 0 then
    
    $\quad$ result := 1
    
    else if e \% 2 = 0 then
    
    $\quad$ half := fastExpo(b, a/2)
    
    $\quad$ result := half $\cdot$ half
    
    else
    
    $\quad$ result := b $\cdot$ fastExpo(b, a-1)
    
    return result\\
    
    Example: $3^{12} = 3^6 \cdot 3^6 = g \cdot g$
    
    Big O is $O(\log n)$
    
    The fastExpo function is an example of "divide-and-conquer".
    
    There are also, for instance, dynamic programming (example: longest common subsequence), and greedy algorithm.
    
    
    
\end{document}