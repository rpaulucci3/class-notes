\documentclass{exam}
\input amssym.tex
\title{CS 2050 Class Notes for 2018-09-14}
\author{Rafael A. Okiishi Paulucci}
\date{}

\begin{document}
	\maketitle
	\section{Section 2.1: Sets}
	
	\textbf{Unordered collection of elements.}
	
	Example: $\{1,2,3\} = \{2,1,3\} = \{1,1,3,2,3,2\}$\newline
	
	$S = \{1,3,7,7,$ Mary, Lee$,3.5,1\}$
	
	The cardinality of $S$ is 6.
	
	$\Bbb Z^+ = \{ 1,2,3,... \}$
	
	The cardinality of $\Bbb Z^+$ is $\infty$.
	
	$B = \{\Bbb Z, \Bbb Z^+, \Bbb R \}$
	
	The cardinality of $B$ is 3.
	
	$C = \{ \{\}, \{1,2\},1,2,\{1,1,2 \}, \{\{\}\} \}$
	
	The cardinality of $C$ is 5. The only element not counted is $\{1,1,2 \}$, because it is equivalent to $\{1,2\}$.
	
	Note: $1 \neq \{1\}$, and $1 \in \{1\}$
	
	Also, the set that contains the empty set is not equal to the empty set itself.
	
	$D = \{\Bbb Z, \Bbb Z^+ \cup \{0\} \cup \Bbb Z^- \}$
	
	The cardinality of $D$ is 1, because the second element is equivalent to the first.\newline
	
	\noindent
	\textbf{Set Builder Notation vs. List Notation}
	
	$B = \{2,4,6,8,...\}$
	
	$\Bbb Z^+ = \{ 1,2,3,4,... \}$
	
	$B = \{ 2x \ | \ x \in \Bbb Z^+\}$ \\
	
	$A = \{1,2\}$
	
	$B = \{5,6,7\}$ 
	
	$A \times B = \{ (1,5),(1,6),(1,7),(2,5),(2,6),(2,7) \}$
	
	$A \times B = \{ (a,b) \ |\ a \in A, b \in B \}$\\
	
	$S = \{2,9,28,65,...\}$
	
	$S = \{ x^3 + 1 \ | \ x \in \Bbb Z^+ \}$\\
	
	$A = B \iff \forall x (x \in A \iff x \in B)$\\
	
	$A \subset B \iff \forall x (x \in A \rightarrow x \in B) \wedge \exists x (x \in B \wedge x \not\in A)$
	
	(Note: the above is a proper subset. A "regular" subset is denoted by $\subseteq$.)
	
	$A \subseteq B \iff \forall x (x \in A \rightarrow x \in B)$

	
\end{document}