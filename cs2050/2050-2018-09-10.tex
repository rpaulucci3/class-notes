\documentclass{exam}
\input amssym.tex

\begin{document}
	\textbf{1. If $n = ab$ then $a \leq \sqrt{n} \lor b \leq \sqrt{n}$ with $n, a, b \in \Bbb Z^+$}
	
	Proof by contraposition:
	
	\begin{tabular}{c|c}
		Step&Rule\\
		\hline
		$\neg(a \leq \sqrt{n} \lor b \leq \sqrt{n})$ & Assume $\neg q$\\
		
		$a \leq \sqrt{n} \wedge b \leq \sqrt{n}$ & DeMorgan's\\
		
		$a > \sqrt{n} \wedge b > \sqrt{n}$ & Definiton of $\leq$\\
		
		$ab > \sqrt{n}\sqrt{n}$ & Multiply LHSs + RHSs\\
		
		$ab > n$ & Multiply\\
	\end{tabular}
	
	Therefore, $ab \neq n$ by definition of $>, =$. Note this is $\neg p$.
	
	Conclusion: I have shown that, assuming $\neg q$ leads to $\neg p$, hence $p \rightarrow q$. \newline
	
	\textbf{2. If $m$ and $n$ are perfect squares, then $mn$ is a perfect square.}
	
	Definition: The integer $x$ is a perfect square if $x = a^2$, with $x, a \in \Bbb Z$.
	
	Direct proof:
	
	\begin{tabular}{c|c}
		Step&Rule\\
		\hline
		$m$ is a perfect square. $n$ is a perfect square. &Assume $p$.\\
		
		$m = a^2, a \in \Bbb Z$&\\
		
		$n = b^2, b \in \Bbb Z$&\\
		
		$mn = a^2 b^2$ &Multiply LHSs + RHSs\\
		
		$mn = (ab)^2$ &Math.\\
	\end{tabular}
	
	Hence $mn$ is a perfect square, by definition, where $mn = c^2$, $c=ab$, with $c \in \Bbb Z$. This is our $q$ proposition.
	
	Conclusion: Assuming $p$ leads to $q$, hence $p \rightarrow q$.\newline
	
	\textbf{3. $\forall x \exists y \exists z (x=y^2 + z^2)$ with $y,z \in \Bbb Z$ and $x \in \Bbb Z^+$}
	
	Proof by counterexample (used to prove a claim is false.)
	
	See $x=3$.
	
	$3 \in \Bbb Z^+$ and yet no sum of $y^2$ and $z^2$ will yield $3$.
	
	Let's look at the smallest perfect squares.
	
	\begin{tabular}{c}
		$0^2 = 0$\\
		$-1^2 = 1^2$ = 1\\
		$-2^2 = 2^2 = 4$\\
	\end{tabular}
	
	And these continue in increasing order.
	
	A sum of 3 could allow some pair of 0 or 1. Yet,
	
	\begin{tabular}{c}
		$0+0=0$\\
		$0+1=1$\\
		$1+0=1$\\
		$1+1=2$\\
	\end{tabular}
	
	Any larger sum overshoots 3.
	
	Conclusion: $x=3$ fails, so this claim is false.\newline
	
	
	\textbf{4. }
	
	Proof by contradiction
	
	Instead of starting from $p \rightarrow q$, we'll start from the following:
	
	$\neg (p \rightarrow q)$
	
	Material implication: $\neg(\neg p \lor q)$
	
	Assume $p \wedge \neg q$. If we find $p \wedge \neg q \wedge q$, there's a contradiction. Likewise for $\neg p \wedge p \wedge \neg q$.
	
\end{document}