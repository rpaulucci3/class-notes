\documentclass{exam}
\usepackage{amssymb,amsmath,listings,parskip}
\begin{document}
	
    \section{5.1: Mathematical induction}
    
    \textbf{"Ladder"}
    
    Can we reach (prove) the first rung? (i.e., the smallest problem size; the basis step, base case, $P(1)$).
    
    Inductive step: $k,\ k+1$ rungs.
    
    $P(k) \rightarrow P(k+1)$. Assume $P(k)$, don't prove it.
    
    Conclusion: since the basis step and induction step have been shown to be true, \textit{by the principle of mathematical induction}, $\forall n P(n), n \in \mathbb Z^+$ is true. \\
    
    $P(n): \sum_{i=1}^{n} i = \frac{n(n+1)}{2}, \forall n P(n), n \in \mathbb Z^+$.
    
    \textbf{Proof by mathematical induction.}
    
    Basis step: $P(1): \sum_{i=1}^{n} i = 1$ The summation (LHS) yields 1.
    
    $\frac{1(1+1)}{2} = 1$ The RHS (closed form) yields 1 when simplified.
    
    Conclusion: since the sum is 1 and the closed form is 1, $P(1)$ is true.
    
    Inductive step:
    
    I will prove $P(k) \rightarrow P(k+1), k \in \mathbb Z^+$.
    
    Assume $P(k): \sum_{i=1}^{k} i = \frac{k(k+1)}{2}$.
    
    \noindent\fbox{
        \parbox{\textwidth}{
            \textit{Aside:}
            
            $1+2+3+4+\dots+k \rightarrow k(k+1)/2$
            
            To add the first $k+1$ numbers, $1+2+3+4+\dots+k+[k+1] \rightarrow k(k+1)/2 + [k+1] = \sum_{i=1}^{k+1} i = \frac{(k+1)[(k+1)+1]}{2}$
        }
    }
    
    $\sum_{i=1}^{k} i + [k+1] = \frac{k(k+1)}{2} + [k+1]$
    
    $\sum_{i=1}^{k+1} i + [k+1] = \frac{k(k+1)+2[k+1]}{2}$ Simplify the summation, find common denominator.
    
    \qquad \qquad$= \frac{(k+1)(k+2)}{2} = \frac{(k+1)((k+1)+1)}{2}$
    
    Note this is our $k+1$ statement. Hence, it does follow from our $k$ statement. I have shown $P(k) \rightarrow P(k+1)$, completing the inductive step.
    
    Conclusion: since the basis step and inductive step are both true, by the principle of mathematical induction, $P(n)$ is true for all positive integers.\\
    
    \newpage
    
    $1+3+5+7+\dots =\ ?,\ n \in \mathbb Z^+$
    
    The sum of the first element is 1.
    
    The sum of the first 2 elements is 4.
    
    The sum of the first 3 elements is 9.
    
    It seems like the sum of the first $n$ elements is $n^2$.
    
    $P(n): \sum_{i=1}^{n} (2i-1) = n^2,\ n \in \mathbb Z^+$ (conjecture)
    
    \textbf{Proof by mathematical induction.}
    
    Basis step: $P(1):$ $\sum_{i=1}^{1} (2i-1) = 1^2$
    
    LHS: $2(1) - 1 = 2-1 = 1$
    
    RHS: $1^2 + 1$
    
    Since LHS = RHS, $P(1)$ is true.
    
    Inductive step:
    
    We will show $P(k) \rightarrow P(k+1), k \in \mathbb Z^+$
    
    $P(k): \sum_{i=1}^{k} (2i-1) = k^2$ Assume $P(k)$.
    
    $\sum_{i=1}^{k} (2i-1) + [2(k+1)-1] = k^2 + [2(k+1)-1]$ Add next term to both
    
    $\sum_{i=1}^{k+1} (2i-1) = k^2 + [2(k+1)-1]$ Clean up sum
    
    \qquad \qquad $= k^2 + 2k + 2 - 1$ Simplify RHS
    
    \qquad \qquad $= k^2 + 2k + 1$ Simplify
    
    \qquad \qquad $= (k+1)^2$ Factor
    
    $P(k)$ does lead to $P(k+1)$, being true, hence, $P(k) \rightarrow P(k+1)$
    
    Conclusion: since the basis step and inductive step are both true, by the principle of mathematical induction, $P(n)$ is true for all positive integers.
    
\end{document}