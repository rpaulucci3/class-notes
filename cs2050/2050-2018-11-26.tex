\documentclass[english]{exam}
\usepackage{babel}
\usepackage{amssymb,amsmath,listings,parskip}
\usepackage{csquotes}
\MakeOuterQuote{"}

\begin{document}
	
    \section{7.1: Finite Probability}
    
    \textit{Experiment:} process yielding one of a set of possible outcomes.
    
    \textit{Sample space:} set of possible outcomes
    
    \textit{Event space:} subset of sample space
    
    \textbf{Socks}
    
    4 orange socks, 5 blue socks in a drawer.
    
    What is the probability of retrieving an orange sock?
    
    P(orange) = $\frac{4}{9}$ = $\frac{/E/}{/S/}$\\
    
    \textbf{Dice}
    
    Roll 2 dice (6-sided)
    
    Rolling $<1,2>$ is different than rolling $<2,1>$. There are 36 possible outcomes.
    
    The possible pairs that add up to 7 are $<1,6>, <2,5>, <3,4>, <4,3>, <5,2>, <6,1>$. There are 6 favourable outcomes.
    
    P(total = 7) = $\frac{6}{36} = \frac{1}{6}$
    
    The possible pair that adds up to 12 is $<6,6>$.
    
    P(total = 12) = $\frac{1}{36}$\\
    
    \textbf{Lottery}
    
    First prize: pick 4 correct digits. Probability is $\frac{1}{10^4}$ (10 ways for each digit)
    
    Second prize: pick 3 out of 4 correct digits (value and placement).
    
    There are 9 ways to "miss" each digit, and 4 possible positions for each digit. Probability is $\frac{36}{10^4}$\\
    
    \textbf{Lottery (again)}
    
    Let $S$ be the set of positive integers from 1 to 40, just like 40 distinct, numbered balls.
    
    We will get a subset of 6 numbers with no replacement. To win the game, you must have the 6 correct numbers.
    
    The probability to win is $\frac{1}{C(40,6)} \approxeq 0.00000026$.
    
    If replacement was allowed, this would become a Stars and Bars problem.
    
    The probability would be $\frac{1}{C(40+6-1, 6)} = \frac{1}{C(45,6)}$\\
    
    \textbf{Poker}
    
    52 cards: 13 ranks $\{2,3,4,5,...,10,J,Q,K,A\}$ $\times$ 4 suits $\{$Spades, Hearts, Clubs, Diamonds$\}$.
    
    Probability("4 of a kind" given 5-hand card):
    
    $$\frac{C(13,1) \cdot C(4,4) \cdot C(48,1)}{C(52,5)} = \frac{13 \cdot 48}{C(52,5)}$$\\
    
    "Full House": 3 of a kind and 2 of a kind (notice that $3Q + 2K \neq 3K + 2Q$)
    
    $$\frac{C(13,1) \cdot C(4,3) \cdot C(12,1) \cdot C(4,2)}{C(52,5)}$$
    
    An alternative way to write the numerator is $P(13,2) \cdot C(4,3) \cdot C(4,2)$.\\
    
    \textbf{Number sequence}
    
    Pick 5 numbers from [1, 50], with no replacement, and order mattering.
    
    Probability: $\frac{1}{P(50,5)}$
    
    \end{document}
