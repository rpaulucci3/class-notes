\documentclass{exam}
\usepackage{amssymb,amsmath,listings,parskip}
\begin{document}
	
    \section{5.3: Recursion}
    
    (1) Functions
    
    (2) Sets
    
    (3) Sequences (strings)\\
    
    Factorials $n!,\ n \in \mathbb Z^{\geq 0}$
    
    Base case: $0! = 1$
    
    Recursive step: $(n+1)! = (n+1) \cdot n!$\\
    
    Fibonacci ($n \in \mathbb Z^{\geq 0}$)
    
    Base case: Fib(0) = 0
    
    Fib(1) = 1
    
    Recursive step: Fib(n+2) = Fib(n) + Fib(n+1)\\
    
    Base case: $3 \in S$
    
    Recursive step:  
    \qquad if $x \in S$ and $y \in S$, then $x+y \in S$
    
    List notation: $S =\{3, 6, 9, 12, 15, \dots\} $
    
    Set builder: $S = \{3x |\ x \in \mathbb Z^+\}$\\
    
    $\Sigma$ is your alphabet. Examples: $\Sigma = \{0,1\}$  or $\Sigma = \{A,T,C,G\}$, and so on.
    
    $\Sigma^*$ (Kleene closure) is the set of all strings that can be built from symbols in $\Sigma$. $\lambda$ is an empty string.
    
    Example: $\Sigma = \{0,1\}$
    
    $\Sigma^* = \{\lambda, 0, 1, 00, 01, 10, 11, 000, 001, 010, 011, 100, 101, 110, 111, 0000, \dots \}$
    
    Recursive definition of $\Sigma^*$.
    
    Base case: $\lambda \in \Sigma^*$
    
    Recursive step: If $w \in \Sigma^*$ and $x \in \Sigma$, then $wx \in \Sigma^*$ (concatenation)
    
    We will notice that the elements 0 and 1, in the previous example, are actually formed by $\lambda0$ and $\lambda1$
    
    $w$ is a string over alphabet $\Sigma$: $w \in \Sigma^*$
    
    Define the reverse, $w^R$, recursively.
    
    Base case: $\lambda^R = \lambda$
    
    Recursive step: If $w \in \Sigma^*$ and $x \in \Sigma$, then $(wx)^R = x(w)^R$ (i.e, put the last character in the front position of the string)
    
    Set of all palindromes over $\Sigma$. Call the set $P$.
    
    Recursive definition: 
    
    Base case: $\lambda \in P$. If $x \in \Sigma, x \in P$.
    
    Recursive step: If $w \in P$ and $x \in \Sigma$, then $xwx \in P$.
    
\end{document}