\documentclass{exam}
\usepackage{amssymb,amsmath,listings,parskip}
\begin{document}
	
    \section{4.2 (cont.)}
    
    $1011011_2 = 133_8 = 5B_{16}$
    
    (convert in blocks of 3 bits for octal, and in blocks of 4 bits for hexadecimal)\\
    
    \textbf{Example:} Convert $241_{10}$ to binary. 
    
    241 div 2 = 120, remainder 1
    
    120 div 2 = 60, remainder 0
    
    60 div 2 = 30, remainder 0
    
    30 div 2 = 15, remainder 0
    
    15 div 2 = 7, remainder 1
    
    7 div 2 = 3, remainder 1
    
    3 div 2 = 1, remainder 1
    
    1 div 2 = 0, remainder 1
    
    Read from bottom to top: $11110001_2$
    
    \section{4.3: Primes, GCD, LCM}
    
    \textbf{Prime:} integer greater than 1 that is divisible by only 1 and itself
    
    Every positive integer $>1$ is divisible at least by 1 and itself.
    
    \textbf{Composite:} positive integer $>1$ that is not prime.
    
    Integer $n$ is composite $\iff \exists a \in \mathbb Z$ s.t. $a | n$ and $1 < a < n$ 
    
    \textbf{Fundamental Theorem of Arithmetic:} every positive integer $>1$ can be written uniquely as a prime number or as the product of two or more primes, where the prime factors are written in non-decreasing order.
    
    Examples: $100 = 2 \cdot 2 \cdot 5 \cdot 5$, $641 = 641$, $999 = 3^3 \cdot 37$
    
    Longer example: $7007$
    
    $2|7007$ fails
    
    $3|7007$ fails
    
    $5|7007$ fails
    
    $7|7007$ works: $7007/7 = 1001$
    
    $7|1001$ works: $1001/7 = 143$
    
    $7|143$ fails
    
    $11|143$ works: $11/143 = 13$
    
    $7007 = 7 \cdot 7 \cdot 11 \cdot 13$
    
    The factorization can be optimized, for example, by only verifying for primes up to the square root of the original number.\\
    
    \textbf{Greatest Common Divisor (GCD)}
    
    gcd(36, 24) = 12
    
    $36 = 2^2 \cdot 3^2$, $24 = 2^3 \cdot 3^1$
    
    What 36 and 24 have in common at most, prime by prime, is $2^2 \cdot 3^1 = 12$ (minimized exponents).
    
    gcd($2^{13} \cdot 3^5 \cdot 7^1 \cdot 13^2,\ 2^2 \cdot 3^1 \cdot 11^5$) = $2^2 \cdot 3^1$
    
     \textbf{Co-prime or relatively prime}
     
     Example: $11^2 \cdot 13^5$ and $3^2 \cdot 5^3$. The gcd is 1.
     
     \textbf{Least Common Multiple (LCM)}
     
     lcm($2^{13} \cdot 3^5 \cdot 7^1 \cdot 13^2, 2^2 \cdot 3^1 \cdot 11^5$) = $2^{13} \cdot 3^5 \cdot 7^1 \cdot 11^5 \cdot 13^2$
     
     lcm(120, 500) = lcm$(2^3 \cdot 3 \cdot 5,\ 2^2 \cdot 5^3) = 2^3 \cdot 3^1 \cdot 5^3 = 3000$\\
     
    gcd(630, 196)
    
    630 mod 196 = 42
    
    gcd(196, 42)
    
    196 mod 42 = 28
    
    gcd(42, 28)
    
    42 mod 28 = 14
    
    gcd(28, 14)
    
    28 mod 14 = 0
    
    gcd(14,0)
    
    So gcd(630, 196) = 14
    
\end{document}