\documentclass{exam}
\input amssym.tex

\begin{document}
	\section{1.7: Intro to Proofs}

	\begin{tabular}{c|c|c}
		Technique&Assume&Show\\
		\hline
		direct proof&$p$&$q$\\
		proof by contraposition&$\neg q$&$\neg p$\\
		vacuous proof&(nothing)&$\neg p$ ($p$ is always false)\\
		trivial proof&(nothing)&$q$($q$ is always true)\\
		proof by contradiction&$p \wedge \neg q$&$?$\\
	\end{tabular} \newline

	Prove p $\rightarrow$ q.
	
	1) Assume p is true.
	
	2) Work in definitions. Do math, do logic.
	
	3) q is then also true.\newline
	
	Definition. An integer $n$ is even if there exists an integer $k$ where $n = 2k$\newline
	
	Definition. An integer $n$ is odd if there exists an integer $k$ where $n = 2k + 1$\newline
	
	\textbf{Direct proof: The sum of two even integers is even.}
	
	$p \rightarrow q$
	a, b $\in {\Bbb Z}$
	
	p: a and b are even
	
	q: a + b is even.
	
	a and b are even (assume p)
	
	a = 2k, k $\in {\Bbb Z}$ (by definition even)
	
	b = 2k', $k' \in {\Bbb Z}$ (by definition even)

	a+b = 2k + 2k' (add LHS, RHS)
	
	a+b = 2(k+k') (factor)
	
	Note: by definition of even, a+b is even, having form a+b = 2k' where k'' = k + k', k'' $\in {\Bbb Z}$
	
	Integers are closed with respect to addition. 
	
	Conclusion: I've shown that assuming $p$ leads to $q$ being true, hence $p \rightarrow q$.\newline
	
	\textbf{Direct proof: If $n$ is odd, then $n^2$ is odd.}
	
	Proof by contraposition: If $n$ is an integer and $3n+2$ is odd, then $n$ is odd.
	
	$p \rightarrow q$
	
	$p: 3n+2$ is odd
	
	$q: n$ is odd ($n \in {\Bbb Z}$)
	
	$\neg$ ($n$ is odd) (assume $\neg q$)
	
	$n$ is even (by definition of even and odd)
	
	$n = 2k, k \in {\Bbb Z}$ (by definition of even)
	
	$3n+2$ Let's examine this.
	
	$3(2k) + 2$ Substitute for $n$.
	
	$2(3k+1)$ Factor
	
	Notice this is $\neg p$, the fact than $3n+2$ is even.
	
	Matching the format of $2k'$, where $k' = 3k+1, k' \in \Bbb Z$
	
	Conclusion: since $\neg q$ leads to $\neg p$, I've shown $p \rightarrow q$.
	
\end{document}