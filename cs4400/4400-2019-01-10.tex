\documentclass[english]{exam}
\usepackage{babel}
\usepackage{amssymb,amsmath,listings,parskip}
\usepackage{csquotes}
\MakeOuterQuote{"}

\begin{document}
	
    \section{Concepts (cont.)}
    
    A DBMS has 3 purposes: to define, manipulate, and provide access to the database.
    
    Database applications have 2 purposes: to query and modify/update the database.
    
    Columns are \textit{attributes}. Rows are \textit{entities}. Table names are \textit{relations}.
    
    DBMS Catalog stores not only the data, but also metadata about them.
    
    \section{Characteristics of the DB approach}
    
    $\bullet$ Data Abstraction: hide the implementation details. Give users a conceptual view of the data and its relations. Allow changes without other levels of conceptualization needing to be aware.
    
    $\bullet$ Self-describing data (metadata).
    
    $\bullet$ Support for multiple views of the data.
    
    \section{Database users}
    
    $\bullet$ Actors of the scene: people who use the DB, including those who create applications for it.
    
    $\bullet$ Workers behind the scene: people who built the DBMS.
    
    \section{Architecture}
    
    A data model has \textit{concepts}, \textit{operations} (transactions, queries, updates) and \textit{constraints} (datatypes and values that are legal; can or can't be null, must have a unique ID etc.)
    
    Data models can be \textit{conceptual}, \textit{physical} or \textit{self-describing}.
    
    Database schema: structure
    
    Database state: a snapshot of the DB at a particular moment in time\\
    
    \textbf{The Three-Schema Architecture: }
    
    $\bullet$ End-users are at the external level (with external views.) 
    
    $\bullet$ Conceptual level: conceptual schema  (first name, last name, address, phone number...)
    
    $\bullet$ Internal level: internal schema (Strings, floating-point numbers, integers, booleans etc.), linked to the binary form of the data stored in memory.
    
    
\end{document}
