\documentclass[english]{exam}
\usepackage{babel}
\usepackage{amssymb,amsmath,listings,parskip}
\usepackage{csquotes}
\MakeOuterQuote{"}

\begin{document}
	
    \section{Entity Relationship Model}
    
    An \textit{entity} exists in a mini-world. An entity relationship (ER) model defines relationships between entities.
     
    Examples: PROFESSOR has 0 or more STUDENT. STUDENT has 0 or more ASSIGNMENT.
     
    These decisions have to be taken before the database is created.
     
    Attributes are characteristics of entities. Each attribute has its \textit{value}, with a certain data type.
    
    Attributes can be simple (e.g.: an ID number), composite (e.g.: an address with number, street, city, state etc.) or multi-valued (e.g.: previous addresses). Attributes can also be calculated.\\
    
    In a diagram, there are shapes for objects: an entity is represented by a rectangle with capital letters; a relationship by a diamond; and an attribute by an oval. \textit{Keys} have underlines under their name. Concentric ovals are multi-valued attributes.

    An entity CAR, for example, can be represented by a specific entity (e.g.: John's car.) The current set of specific entities in a given type is the state of the entities.\\
    
    Each edge in the diagram is read "name $\rightarrow$ line $\rightarrow$ relationship $\rightarrow$ number." If lines are not double, we read "may take..."
    
    A \textit{weak entity} cannot be distringuished by its attributes. It must have an identifying relationship.
    
    Relations can be recursive.
    
    \textit{Discriminators} are not strong enough to be a key.
     
\end{document}
