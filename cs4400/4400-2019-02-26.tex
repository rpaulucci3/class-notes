\documentclass[english]{exam}
\usepackage{babel}
\usepackage{amssymb,amsmath,listings,parskip}
\usepackage{csquotes}
\MakeOuterQuote{"}

\begin{document}
	
    \section{Normalization}
    
    Normalization is the process of improving a relation by analyzing it usnig its functional dependencies and primary keys.
    
    We attempt to get a relation into a higher normal form.
    
    $\bullet$ First Normal Form (1NF) - more relaxed requirements
    
    \textit{"The key"}: The attributes must include only atomic (indivisible) values (i.e.: no combination values, no multi-values.) The primary key must be able to reach all of the values.
    
    $\bullet$ Second Normal Form (2NF)
    
    \textit{"The whole key"}: For every non-prime attribute $A \in R$, $A$ is fully functionally dependent on the primary key of $R$.
    
    $\bullet$ Third Normal Form (3NF)
    
    \textit{"Nothing but the key"}: None of the attributes of the relation are determined via a transitive relationship.
    
    $\bullet$ Boyce-Codd Normal Form, 4NF and 5NF - more stringent requirements
     
\end{document}
