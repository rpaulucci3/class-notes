\documentclass[english,openany]{book}
\usepackage[utf8]{inputenc}
\usepackage{babel}
\usepackage{amssymb,amsmath,listings,parskip}
\usepackage{csquotes}
\MakeOuterQuote{"}

\begin{document}

    \title{Some notes from CS 3790: Introduction to Cognitive Science}
    \author{Rafael A. O. Paulucci}
    \date{Fall 2019}

    \maketitle

    \tableofcontents

    \lstset{frame=tb,
    showstringspaces=false,
    columns=flexible,
    basicstyle={\small\ttfamily},
    numbers=none,
    breaklines=true,
    breakatwhitespace=true,
    tabsize=4
    }

    \chapter{Introduction}


\section{Mental Representations}

$\bullet$ The aim of cognitive science is to explain how a mind, human or otherwise, accomplishes daily tasks.

$\bullet$ For example, how does a dog know that it's supposed to meet its owner at the door, with a "sad" face, when it breaks something in the house? How can a marketing specialist determine factors that will bring in new customers?

$\bullet$ Since just looking at a brain provides no answers for those questions, metaphors, analogies and other forms of study are required to assist in understanding the mind.

$\bullet$ Is thinking like computing? Is computing like thinking? We don't know for sure.

$\bullet$ When sensing a landscape (by looking at it), we "decode" elements we see (we know what a sky, road, hill, vineyard etc. look like.) We also ignore what is not useful for analyzing a scene, given the context.

$\bullet$ Perhaps for evolutionary reasons, humans tend to easily recognize faces, as well as attempt to see them in inanimate patterns.

$\bullet$ A non-mental representation is something that can be mapped physically, such as a map.

$\bullet$ A mental representation is a "stand-in" for something that is not in the mind. The world is "outside" the mind. Representations attempt to use some form of "truth" to bring it "inside". Parts of it are rules, concepts, images and analogies.

$\bullet$ Different representations require different procedures. For example, one could ask you to perform a numerical division in Roman numerals. Since most people are used to Arabic numerals, such a calculation would require decoding Roman into Arabic, performing the calculation (in Arabic), and encoding the result back into Roman.\\

\section{History of Cognitive Science}

$\bullet$ Plato: Knowledge is reminiscence. What is known are innate truths.

$\bullet$ Aristotle: Knowledge is rules learned from sensory experience. What is known is empirically derived. "Mind is an unscribed tablet (\textit{tabula rasa)}."\\

$\bullet$ Rationalism: Descartes and Leibniz. Some things are known by intuition; others are deduced by intuition. Some truths are innate.

$\bullet$ Empiricism: Hume and Locke. Our source of knowledge in a subject or its concepts comes strictly from sense experience.

$\bullet$ Empiricism + Rationalism: Immanuel Kant. Faculties of the mind involved in knowing: sensibility and understanding. "Thoughts without concepts are empty."\\

$\bullet$ Experimental psychology: Wilhelm Wundt. Study of mental processes systematically. Knowledge comes from synthesizing representations from what we sense or think. We know when an experience begins and ends.

$\bullet$ Structuralism: Edward Titchener. Study of the structure of the conscious mind. Focus on sensations, images and feelings. "Taxonomy" (inspired by biology) of the structures of mind.

$\bullet$ Functionalism: William James. Influenced by Darwin. Measure functions of mind in terms of adaptation to environment.

$\bullet$ Behaviorism: Watson and Skinner. Only concerned with observable responses to stimuli. (Dominated Western psychology in the early-mid 20th century.)\\

$\bullet$ Computationalism: George Miller. Is the mind like a computer? Humans have mental capabilities for encoding and decoding. Capacity for thinking and short-term memory are limited, but that limitation can be mitigated by encoding information into chunks.

$\bullet$ Dartmouth Conference (1956): John McCarthy, Marvin Minsky, Allen Newell and Herbert Simon. Early artificial intelligence.

$\bullet$ Noam Chomsky: Language is not a learned habit, but an acquired one. We have mental grammars, and an inherited mechanism to acquire languages.

\section{Elements of Cognitive Science}

$\bullet$ Behavior experiments, computational experiments, grammars and patterns in languages, brain imaging (ablation experiments)

\end{document}
