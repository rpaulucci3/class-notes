\documentclass[english,openany]{book}
\usepackage[utf8]{inputenc}
\usepackage{babel}
\usepackage{amssymb,amsmath,listings,parskip}
\usepackage{csquotes}
\MakeOuterQuote{"}
\usepackage{hyperref}
\hypersetup{
    colorlinks,
    allcolors=black,
    hidelinks
}

\begin{document}

    \title{Some notes from CS 3790: Introduction to Cognitive Science}
    \author{Rafael A. O. Paulucci}
    \date{Fall 2019}

    \maketitle

    \tableofcontents

    \lstset{frame=tb,
    showstringspaces=false,
    columns=flexible,
    basicstyle={\small\ttfamily},
    numbers=none,
    breaklines=true,
    breakatwhitespace=true,
    tabsize=4
    }

    \chapter{Introduction}


\section{Mental Representations}

$\bullet$ The aim of cognitive science is to explain how a mind, human or otherwise, accomplishes daily tasks.

$\bullet$ For example, how does a dog know that it's supposed to meet its owner at the door, with a "sad" face, when it breaks something in the house? How can a marketing specialist determine factors that will bring in new customers?

$\bullet$ Since just looking at a brain provides no answers for those questions, metaphors, analogies and other forms of study are required to assist in understanding the mind.

$\bullet$ Is thinking like computing? Is computing like thinking? We don't know for sure.

$\bullet$ When sensing a landscape (by looking at it), we "decode" elements we see (we know what a sky, road, hill, vineyard etc. look like.) We also ignore what is not useful for analyzing a scene, given the context.

$\bullet$ Perhaps for evolutionary reasons, humans tend to easily recognize faces, as well as attempt to see them in inanimate patterns.

$\bullet$ A non-mental representation is something that can be mapped physically, such as a map.

$\bullet$ A mental representation is a "stand-in" for something that is not in the mind. The world is "outside" the mind. Representations attempt to use some form of "truth" to bring it "inside". Parts of it are rules, concepts, images and analogies.

$\bullet$ Different representations require different procedures. For example, one could ask you to perform a numerical division in Roman numerals. Since most people are used to Arabic numerals, such a calculation would require decoding Roman into Arabic, performing the calculation (in Arabic), and encoding the result back into Roman.\\

\section{History of Cognitive Science}

$\bullet$ Plato: Knowledge is reminiscence. What is known are innate truths.

$\bullet$ Aristotle: Knowledge is rules learned from sensory experience. What is known is empirically derived. "Mind is an unscribed tablet (\textit{tabula rasa)}."\\

$\bullet$ Rationalism: Descartes and Leibniz. Some things are known by intuition; others are deduced by intuition. Some truths are innate.

$\bullet$ Empiricism: Hume and Locke. Our source of knowledge in a subject or its concepts comes strictly from sense experience.

$\bullet$ Empiricism + Rationalism: Immanuel Kant. Faculties of the mind involved in knowing: sensibility and understanding. "Thoughts without concepts are empty."\\

$\bullet$ Experimental psychology: Wilhelm Wundt. Study of mental processes systematically. Knowledge comes from synthesizing representations from what we sense or think. We know when an experience begins and ends.

$\bullet$ Structuralism: Edward Titchener. Study of the structure of the conscious mind. Focus on sensations, images and feelings. "Taxonomy" (inspired by biology) of the structures of mind.

$\bullet$ Functionalism: William James. Influenced by Darwin. Measure functions of mind in terms of adaptation to environment.

$\bullet$ Behaviorism: Watson and Skinner. Only concerned with observable responses to stimuli. (Dominated Western psychology in the early-mid 20th century.)\\

$\bullet$ Computationalism: George Miller. Is the mind like a computer? Humans have mental capabilities for encoding and decoding. Capacity for thinking and short-term memory are limited, but that limitation can be mitigated by encoding information into chunks.

$\bullet$ Dartmouth Conference (1956): John McCarthy, Marvin Minsky, Allen Newell and Herbert Simon. Early artificial intelligence.

$\bullet$ Noam Chomsky: Language is not a learned habit, but an acquired one. We have mental grammars, and an inherited mechanism to acquire languages.

\subsection{Elements of Cognitive Science}

$\bullet$ Material sourced from behavior experiments, computational experiments, grammars and patterns in languages, brain imaging (ablation experiments)...

\chapter{Comparing Mind to a Program}

\section{C.R.U.M.}

\subsection{Definitions}

$\bullet$ \textbf{C.R.U.M.: Computational -- Representational Understanding of Mind.} The Central Hypothesis: thinking can best be understood in terms of representational structures of the mind, and computational procedures that operate on those structures.

$\bullet$ Logical structures: data structures and algorithms are like representations and procedures. Running a program is like thinking.

$\bullet$ Physical structures: neurons and connections for representations are like the "hardware". Neuron firing and activation for procedures are like processors running algorithms.

$\bullet$ Three-way metaphor: Learning something influences how an algorithm could be designed. Learning how a physical structure works influences the understanding of its logical use.

\subsection{Cognitive Science Model}

$\bullet$ Theory, model, program, platform.

$\bullet$ Cognitive theory: set of representational structures and set of processes that operate on them.

$\bullet$ Computational model: interpret structures and processes by analogy to computing, its data structures and algorithms.

$\bullet$ Program: implement the model in a programming language.

$\bullet$ Platform: the "hardware".

\subsection{Examples}

\subsubsection{Learning to add}

\begin{enumerate}
    \item Piles of things: counting real-world, tangible objects.

    \item Number system: $1, 2, 3, 4, 5, 6...$

    \item Define concepts such as "larger than" (i.e., in the number system, what appears after is larger; what appears before is not larger)
    
    \item Map piles of things to this number system. Get things from a pile until they run out, thus mapping the amount of things to a number. (Mistakes such as skipping, double-counting, incorrectly saying a number etc. happen until enough practice has been done.)'
    
    \item Representation: $1 = \bullet,\ 2 = \bullet \bullet,\ 3 = \bullet \bullet \bullet$. Define the "size of a pile."
    
    \item Define a symbol for the addition operation: $+$\\ Make it tangible: $\bullet \bullet + \bullet \bullet \bullet = \bullet \bullet \bullet \bullet \bullet$\\
    The idea of "add" is now the same as "combine piles of things."
    
    \item Introduce the concept of bigger numbers, that cannot be so easily represented in piles: $13 + 28 = ?$
    
    \item Combine and carry operations: considering each number individually, $(1, 3) + (2, 8) = (3, 11)$. Then, carry: $(3, 11) = (4, 1)$, which is the same as $30 + 11 = 41$.
\end{enumerate}

$\bullet$ Representations: how are numbers represented, especially when we get to numbers larger than $9$?

$\bullet$ Addition could also have been represented as a computer program, with functions such as \textit{vectorize}, \textit{carry}, \textit{reduce} etc.

\subsubsection{Ordering food from a menu}

$\bullet$ Understanding how each word on the menu maps to a tangible ingredient

$\bullet$ Considering the environment/culture you are in, past experiences in ordering (e.g.: McDonald's burger is different from homemade burger; the restaurant I'm in has served me salty food in the past...)

$\bullet$ Making decisions about the price vs. value of each item

$\bullet$ Deciding the most appropriate form of ordering, and encoding the phrase to be said (or the buttons to be pressed.)

\subsection{Iterative Development}

$\bullet$ Theory, model and program affect one another.

$\bullet$ Value of a working program: does it scale (i.e., is it computationally reasonable)? Is it consistent with observed behavior (psychologically plausible)? Can it predict (human) behavior?

\subsection{Evaluation of the Cognitive Model}

Logic, Rules, Concepts, Analogies, Images, Connections.

\subsubsection{Evaluation criteria}

$\bullet$ Representational power: capabilities and efficiency in representing real-world concepts and objects.

$\bullet$ Computational power: abilities in problem solving, learning and language.

$\bullet$ Psychological plausibility: capabilities in accounting for human cognition qualitatively and quantitatively.

$\bullet$ Neurological plausibility: consistency with neurological observations and knowledge of neural architecture. 

$\bullet$ Practical applicability: how can it be applied to practical situations (e.g.: teaching, driving, making business decisions...)?

\chapter{Logic}

\section{A Brief History}

$\bullet$ Do people think logically? When do they do so? And about what?

$\bullet$ Is logic the backbone of our thinking?

\subsection{Syllogism}

$\bullet$ Aristotle: syllogism (two premises $\therefore$ one conclusion.)

$\bullet$ Possible premises: all A are B; all A are not B.

$\bullet$ Contingent premises: maybe all A are B; maybe all A are not B; some A are B, some A are not B. \textbf{Those are not allowed under this system.}

$\bullet$ Example: ("All X are Z", "Y is X") $\therefore$ "Y is Z"

$\bullet$ The form admits analysis. What is important is the form, not the content.

\subsection{Logicism}

$\bullet$ Gottlob Frege, Bertrand Russell: all of mathematics is reducible to logic.

$\bullet$ Symbols, axioms, rules of inference, and terms such as "all", "some", "class", "relation".

\subsection{Computationalism}

$\bullet$ Alonzo Church, Alan Turing: what's computable? What's effectively calculable?

$\bullet$ John McCarthy: Artificial Intelligence based on logic.

\section{Formal Logic}

\subsection{Propositional Logic}

$\bullet$ Propositions can be true of false

$\bullet$ Letters represent atomic statements (e.g.: P: is it raining; Q: it is cloudy.)

$\bullet$ $\neg P$: not P, $P \rightarrow Q$: if P then Q, $P \lor Q$: P or Q, $P \wedge Q$: P and Q

\subsection{Modal Logic}

$\bullet$ Keith is happy vs. Keith is \textit{usually} happy

$\bullet$ Qualifying propositions: truth (necessarily, possible, impossible), deontic (obligatory, permissible), epistemic (known), doxastic (believed).

\subsection{Predicate Logic}

$\bullet$ Keith is a professor vs. \texttt{is-professor(Keith)}

$\bullet$ Predicates act like truth functions: \texttt{teaches(Keith, CS3790)}

$\bullet$ Quantifiers apply to variables ($\forall, \exists...$)

$\bullet$ Probability functions $P(x)$ assign a value to the truth of a statement $x$, between 0 (false) and 1 (true.)

\subsection{Inference}

% TODO: possibly add example for modus tollens; rules of generalization 

$\bullet$ \textit{Modus ponens}: "affirms by affirming". 
\\If it's raining, then it's cloudy.\\ It's raining.\\ $\therefore$ It's cloudy.

$\bullet$ \textit{Modus tollens}: "denies by denying".

\newpage
$\bullet$ Universal instantiation:\\ All meerkats are mammals.\\ Ralph is a meerkat.\\ $\therefore$ Ralph is a mammal.

\subsubsection{Bayes' Rule}

$\bullet$ Probability of an outcome is determined by a prior probability and likelihood based on observed data.

$$P (\textrm{hypothesis } | \textrm{ evidence}) = \frac{P(\textrm{evidence } | \textrm{ hypothesis}) \cdot P(\textrm{hypothesis})}{P(\textrm{evidence})}$$

% TODO: add examples below

\subsubsection{Deductive Inference}

$\bullet$ From general to specific.

\subsubsection{Inductive Inference}

$\bullet$ From specific to general.

% Simple induction, analogical induction

\subsubsection{Abductive Inference}

$\bullet$ From incomplete to best prediction.

\subsection{Examples}

\subsubsection{Wason Selection Task (Four-card Problem)}

$\bullet$ "If a card has a vowel on one side, then it has an even number on the other side." Which cards must be turned to test this hypothesis?

$\bullet$ Example: cards $A, B, 4, 7$. We flip $A$ and $7$.

$\bullet$ People usually get a wrong answer, but binding cards to real-world situations (e.g.: drinking age, other social rules) sharply increases the success rate.

$\bullet$ Contrapositive: "If a card has an odd number on one side, then it cannot have a vowel on the other side."

\end{document}
