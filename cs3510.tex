\documentclass[english,openany]{book}
\usepackage[utf8]{inputenc}
\usepackage{babel}
\usepackage{amssymb,amsmath,listings,parskip}
\usepackage{csquotes}
\MakeOuterQuote{"}

\begin{document}

    \title{A few notes from CS 3510: Design \& Analysis of Algorithms}
    \author{Rafael A. O. Paulucci}
    \date{Spring 2019}

    \maketitle

    \tableofcontents

    \lstset{frame=tb,
    showstringspaces=false,
    columns=flexible,
    basicstyle={\small\ttfamily},
    numbers=none,
    breaklines=true,
    breakatwhitespace=true,
    tabsize=4
    }

    \chapter{Algorithms}

    General idea: a set of computational instructions written with if, for, memory access.

    \section{Lost Cow problem}

    \begin{lstlisting}
    for i = 1, 2, ...
        move left i
        move right 2i
        move left i
        ...
    \end{lstlisting}

    If gate is at $x$, we will reach it at step $|x|$.

    Cost: $\sum_{i=1}^{|x|} 4i = \frac{4|x|(|x|+1)}{2} = 2|x|(|x|+1)$ $\in O(|x|^2)$

    $f(n) \in O(g(n))$ if there exist $c, n_0$ s.t. $\forall n \geq n_0$, $f(n) \leq c \cdot g(n)$. (i.e., if $\lim\limits_{n \rightarrow \infty}$ $\frac{f(n)}{g(n)} \leq c$)

    $n \in O(n^2) \iff \lim\limits_{n \rightarrow \infty} \frac{n}{n^2} = \lim\limits_{n \rightarrow \infty} \frac{1}{n} = 0$

    $f(n) \in \Omega(g(n)) \iff g(n) \in O(f(n))$

    $f(n) \in \theta(g(n)) \iff  f(n) \in O(g(n)) \wedge g(n) \in O(f(n))$\\

    \newpage
    More efficient version:

    \begin{lstlisting}
    for i = 1, 2, ...
        move left 2**i
        move right 2*(2**i)
        move left 2**i
        ...
    \end{lstlisting}

    Correctness: once $2^i > |x|$, we are done.\\

    \section{Fibonacci}

    \begin{lstlisting}
    def recursiveFib(n):
        if n <= 1
            return 1
        else
           return recursiveFib(n-1) + recursiveFib(n-2)
    \end{lstlisting}

    Runtime to compute $f(n)$: $\theta(f(n)) \rightarrow \theta(\frac{1+\sqrt{5}}{2})^n \in \Omega(1.618^n)$\\

    \subsection{Iterative Fibonacci}

    \begin{lstlisting}
    def iterFib(n):
        a[0] = 1
        a[1] = 1
        for i = 2 to n
            a[i] = a[i-1] + a[i-2]
        return a[n]
    \end{lstlisting}

    Number of digits is $\theta(\log_2 (1.618^n))$


    \subsection{Fibonacci addition}

    \textit{Theorem:} $Fib(n) = \left(  \frac{1+\sqrt{5}}{2}  \right) ^n + \left(  \frac{\sqrt{5} - 1}{2}  \right) ^n \approx 1.6^n + 0.6^n$

    $\log_2 Fib(n) \approx \log_2 1.6^n \approx (\log_2 1.6) \cdot n \in \theta(n)$

    $Fib(49) =\ 7778742049$

    $Fib(50) = 12586269025$

    Numbers are stored in little-endian form, as an array.

    (a, b, c below are arrays for digits in little-endian)

    \begin{lstlisting}
    def add(a, b):
        for i = 1 to max(a.length, b.length):
            c[i] = a[i] + b[i]

    def carry(c):
        for i = 1 to c.length:
            c[i + 1] += c[i] / 10
            c[i] %= 10
    \end{lstlisting}

    This runs in $O(n^{1.618})$, but can be optimized.

    \newpage

    \section{Multiplication algorithms}

    \textit{Theorem:} we can multiply two $n$-digit numbers in $O(n^{\log_2 3})$ time.

    \subsection{Naive algorithm}

    Every $d$-digit number in base 10 is actually a sum of polynomials in base 10.

    $n = \sum_{i=0}^{d} 10^i$ a[$i$]

    \begin{lstlisting}
    def naiveMultiply(a, b):
        for i = 0 to a.length:
            for i = 0 to b.length:
                c[i + j] += a[i] * b[j]
        carry(c)
    \end{lstlisting}

    a[d] ... a[0] = a[d] ... a[d/2]$ \cdot 10^{d/2}$ + a[d/2 - 1] - a[0]

    So $1234 \cdot 5678 = (12 \cdot 100 + 34)(56 \cdot 100 + 78) = 12 \cdot 56 \cdot 10^4 + 34 \cdot 78 + 12 \cdot 78 \cdot 100 + 34 \cdot 76 \cdot 100$

    \subsection{Karatsuba's algorithm}

    $A^\uparrow \times B^\downarrow + A^\downarrow \times B^\uparrow = (A^\uparrow + A^\downarrow) \times (B^\uparrow \times B^\downarrow) -A^\uparrow \times B^\uparrow - A^\downarrow \times B^\downarrow$

    Assume number of digits is $d$.

    \begin{lstlisting}
    def multiply(a, b):
        x = multiply(a[d] to a[d/2], b[d] to b[d/2])
        y = multiply(a[d/2 - 1] to a[0], b[d/2 - 1] to b[0])
        z1 = (a[d] to a[d/2]) + (a[d/2 - 1] to a[0])
        z2 = (b[d] to b[d/2]) + (b[d/2 - 1] to b[0])
        z = multiply(z1, z2) - x * y
        return x * (10**d) + y + z * (10**(d/2))
    \end{lstlisting}

    Runtime recurrence: $T(d) \leq 3 \cdot T(d/2) + O(d)$

    The recursive tree has $L = \log_2 d$ layers. The number of calls on the bottom level will be $3^L \leq 3^{\log_2 d}$

    If there are fewer than 10 layers, it is advisable to switch to the naive algorithm.

\end{document}
