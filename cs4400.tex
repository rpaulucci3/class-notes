\documentclass[english,openany]{book}
\usepackage[utf8]{inputenc}
\usepackage{babel}
\usepackage{amssymb,amsmath,listings,parskip}
\usepackage{csquotes}
\MakeOuterQuote{"}

\begin{document}

    \title{Some notes from CS 4400: Introduction to Database Systems}
    \author{Rafael A. O. Paulucci}
    \date{Spring 2019}

    \maketitle

    \tableofcontents

    \lstset{frame=tb,
    language=SQL,
    showstringspaces=false,
    columns=flexible,
    basicstyle={\small\ttfamily},
    numbers=none,
    breaklines=true,
    breakatwhitespace=true,
    tabsize=4
    }

    \chapter{Introduction}

    \section{Concepts}

    $\bullet$ Data: known facts that can be recorded.

    $\bullet$ Database: a collection of \textbf{related} data.

    $\bullet$ Database Management System (DBMS): software package for building and maintaining databases.

    $\bullet$ Database System (DBS): the DBMS, the data itself, and the applications.

    A DBMS has 3 purposes: to define, manipulate, and provide access to the database.

    Database applications have 2 purposes: to query and modify/update the database.

    Columns are \textit{attributes}. Rows are \textit{entities}. Table names are \textit{relations}.

    DBMS Catalog stores not only the data, but also metadata about them.

    \section{Characteristics of the database approach}

    $\bullet$ Data Abstraction: hide the implementation details. Give users a conceptual view of the data and its relations. Allow changes without other levels of conceptualization needing to be aware.

    $\bullet$ Self-describing data (metadata).

    $\bullet$ Support for multiple views of the data.

    \subsection{Database users}

    $\bullet$ Actors of the scene: people who use the DB, including those who create applications for it.

    $\bullet$ Workers behind the scene: people who built the DBMS.

    \subsection{Architecture}

    A data model has \textit{concepts}, \textit{operations} (transactions, queries, updates) and \textit{constraints} (data types and values that are legal; can or can't be null, must have a unique ID etc.)

    Data models can be \textit{conceptual}, \textit{physical} or \textit{self-describing}.

    Database schema: structure

    Database state: a snapshot of the DB at a particular moment in time\\

    \textbf{The Three-Schema Architecture: }

    $\bullet$ End-users are at the external level (with external views.)

    $\bullet$ Conceptual level: conceptual schema  (first name, last name, address, phone number...)

    $\bullet$ Internal level: internal schema (Strings, floating-point numbers, integers, booleans etc.), linked to the binary form of the data stored in memory.

    \section{Terminology}

    SQL stands for Structured Query Language. It is relational, not imperative.

    Sample command:

    \begin{lstlisting}
    SELECT attribute_list FROM table_list WHERE condition
    \end{lstlisting}

    The symbols we can use for are $<, >, <=, >=, +, <>$ (not equal to)

    $*$ is a wildcard.

    Other keywords include OR, AND, DISTINCT, LIKE, BETWEEN (inclusive)

    $\%$ matches 0 or more characters. $\_$ matches exactly 1 character.

    \chapter{Relational Algebra}

    \section{Relational Algebra}

    SQL is a standard. MySQL doesn't implement all the standards, but includes extensions.

    Relational algebra is a family of algebras used for modelling the data stored in relational databases, and defining queries on it.\\

    We can have a command like SELECT * FROM

    WHERE (filter out rows)

    GROUP BY

    HAVING (filter out groups)

    ORDER BY\\

    \textbf{Aggregate functions:} count, sum, min, max, avg\\

    \textbf{Relational algebra symbols:}

    select: $\sigma$

    project: $\pi$

    rename: $\rho$

    order by: $\tau$

    group by: $\gamma$

    natural join: $\bowtie$

    Cartesian product: $\times$\\

    Example: $\pi$ fname, lname ($\sigma$ fname = 'john', lname = 'smith' EMPLOYEE)\\

    \textbf{Set operations}

    $\cup$ union, $\cap$ intersection, $-$ difference, $\times$ cross product

    \chapter{Operation Examples}

    \section{Nulls and comparison operators}

    A null value in SQL can mean not available, withheld or not applicable.\\

    \textbf{Comparison operators with nested queries}

    IN, ANY, SOME, ALL, UNION.

   \section{ORDER BY and CONCAT}

    We can use ORDER BY \text{attribute\_list} ASC (ascending, which is the default) or DESC (descending)

    We can use CONCAT to concatenate multiple columns

    \section{UNION}

    UNION works similar to OR, but assumes DISTINCT and does not select null values. It also requires both queries to select the same number of columns.

    \chapter{Entity Relationship Model}

    \section{Terminology}

    An \textit{entity} exists in a mini-world. An entity relationship (ER) model defines relationships between entities.

    Examples: PROFESSOR has 0 or more STUDENT. STUDENT has 0 or more ASSIGNMENT.

    These decisions have to be taken before the database is created.

    Attributes are characteristics of entities. Each attribute has its \textit{value}, with a certain data type.

    Attributes can be simple (e.g.: an ID number), composite (e.g.: an address with number, street, city, state etc.) or multi-valued (e.g.: previous addresses). Attributes can also be calculated.\\

    A \textit{weak entity} cannot be distinguished by its attributes. It must have an identifying relationship.

    Relations can be recursive.

    \textit{Discriminators} are not strong enough to be a key.

    \section{ER Diagram Rules}

    In a diagram, there are shapes for objects: an entity is represented by a rectangle with capital letters; a relationship by a diamond; and an attribute by an oval. \textit{Keys} have underlines under their name. Concentric ovals are multi-valued attributes.

    An entity CAR, for example, can be represented by a specific entity (e.g.: John's car.) The current set of specific entities in a given type is the state of the entities.\\

    Each edge in the diagram is read "name $\rightarrow$ line $\rightarrow$ relationship $\rightarrow$ number." If lines are not double, we read "may take..."

    \section{Enhanced Entity Relationship Model}

    Adds generalization and specialization to the ER diagram.

    EER adds the ability to model entity subclasses and superclasses, and allows for inherited attributes and relationships.

    Example: EMPLOYEE has subclasses SECRETARY, ENGINEER that inherit from EMPLOYEE.

    \textit{Specialization circles} can be disjoint ($d$) or overlapping ($d$).

    \chapter{Normalization}

    \section{Normal Forms}

    Normalization is the process of improving a relation by analyzing it using its functional dependencies and primary keys.

    We attempt to get a relation into a higher normal form.

    $\bullet$ First Normal Form (1NF) - more relaxed requirements

    \textit{"The key"}: The attributes must include only atomic (indivisible) values (i.e.: no combination values, no multi-values.) The primary key must be able to reach all of the values.

    $\bullet$ Second Normal Form (2NF)

    \textit{"The whole key"}: For every non-prime attribute $A \in R$, $A$ is fully functionally dependent on the primary key of $R$.

    $\bullet$ Third Normal Form (3NF)

    \textit{"Nothing but the key"}: None of the attributes of the relation are determined via a transitive relationship.

    $\bullet$ Boyce-Codd Normal Form, 4NF and 5NF - more stringent requirements

\end{document}
